\documentclass[10pt,a5paper]{article}

% === PAQUETES === (((
\usepackage[utf8]{inputenc}
\usepackage{amsfonts}
\usepackage{amsmath}
\usepackage{graphicx}
\usepackage{anysize} 
% )))

% === DATOS === (((
% \pdfinfo{%
	% /Title    (Tarea <++> de Ecuaciones Diferenciales)
	% /Author   (Sandra Elizabeth Delgadillo Alemán)
% }
\marginsize{1cm}{1cm}{1cm}{1cm} 
\pagestyle{empty}
% )))

% === TITULO === (((
\newcommand{\titulo}{
	\begin{minipage}{0.25\linewidth}
		\includegraphics[width= 0.9 \linewidth]{IMAGENES/log23.png}
	\end{minipage}
	\begin{minipage}{0.75\linewidth}
		\begin{center}
			\bfseries
			CENTRO DE CIENCIAS BÁSICAS \\
			DEPARTAMENTO DE MATEMÁTICAS Y FÍSICA \\
			ACADEMIA DE MATEMÁTICA AVANZADAS
		\end{center}
	\end{minipage}\\
	\begin{table}[ht]
		\centering
		\begin{tabular}{|*{3}{l|}p{3cm}|}
			\hline
			\textbf{Nombre del Estudiante:} & & \textbf{Fecha:} & \\ \hline
			\textbf{Materia:} & Ecuaciones Diferenciales & \textbf{Carrera:} &  \\ \hline
			\textbf{Profesor:} & Sandra Elizabeth Delgadillo Alemás & \textbf{Semestre:} & \\ \hline
			\textbf{Periodo:} & () Enero--Junio () Agosto--Diciembre & & \\ \hline
			\textbf{Tipo de Examen:} & Parcial: 1() \hspace{2mm} 2() \hspace{2mm} 3() & \textbf{Calificación:} & \\ \hline
		\end{tabular}
	\end{table}
} 
% )))

\begin{document}

\titulo

\begin{enumerate}
	\item \textit{Calsifica las siguientes ecuaciones diferenciales según su tipo, orden y linealidad.} \\[2mm]
		\textbf{Solución.} 
		\begin{table}[ht]
			\centering
			\begin{tabular}{|*{6} {l|}}
				\hline
				Ec. & Tipo & Orden & Lineal & V.S. \\ \hline
				\(\sqrt{1-x} \cdot \dfrac{d^2y}{dx^2} +2x \cdot \dfrac{dy}{dx} = y\) & Ordinaria & \(2^{do}\) Orden & Lineal & No \\ \hline
				\(y'' -t^2yy' = \sin t\) & Ordinaria & \(2^{do}\) Orden & No lineal & No. \\ \hline
			\end{tabular}
		\end{table}
	\item \textit{Verifica que la función \(y(x) = 3 \sin 2x+e^{-x}\) es solución explícita de la ecuación diferencial} 
		\[
			y'' +4y=5e^{-x}.
		\]
		\textbf{Solución.} Tenemos que
		\[
			\begin{array}{rcl}
				y' & = & 6 \cos 2x-e^{-x} \\[2mm]
				y'' & = & -12 \sin 2x+e^{-x}
			\end{array}
		\]
		Así que
		\[
			(-12 \sin 2x+e^{-x}) +4(3 \sin 2x+e^{-x}) = 5e^{-x}.
		\]
	\item \textit{Compruebe que \(y- \ln y=x^2+1\) es solución general de la ecuación diferencial}
		\[
			y' = \dfrac{2xy}{y-1}.
		\]
		\textbf{Solución.} Tenemos que
		\[
			\begin{array}{rcl}
				y' +2 \dfrac{y'}{y} =2x & \;\implies\; & y' \Bigg(1+ \dfrac{2}{y}\Bigg) = 2x \\[2mm]
				& \;\implies\; & y' = \dfrac{2x}{1+2/y} = \dfrac{2xy}{y+2}. \qed 
			\end{array}
		\]
	\item \textit{Determina si las siguientes ecuciones diferneciales son de V.S. o lineales y resuélvelas con el método correspondiente. En cada inciso, indica cuál es solución general implícita y encuentra la explícita.} 
		\begin{enumerate}
			\item \(x \cdot \dfrac{dy}{dx} -2y=x^3e^{-x} -5x\). \\[2mm]
				\textbf{Solución.} \fbox{Es lineal. No es de V.S.} \\[2mm]
				Tenemos que
				\[
					\dfrac{dy}{dx} - \dfrac{2}{x} y = x^2e^{-x} -5.
				\]
				Entocnes \(p(x) = - \dfrac{2}{x}\), luego
				\[
					\mu (x) = e^{\int p} = e^{\int -2/x} = e^{-2  \log x} =e^{\log x^{-2}} =x^{-2}.
				\]
				Por tanto,
				\[
					\begin{array}{rcl}
						x^{-2} \dfrac{dy}{dx} - \dfrac{2}{x^3} y & = & e^{-x} -5x^{-2} \\[5mm]
						\dfrac{d}{dx} \Bigg(x^{-2} y\Bigg) & = & e^{-x} -5x^{-2} \\[5mm]
						\dis\int \dfrac{d}{dx} \Bigg(x^{-2} y\Bigg) dx & = & \dis\int (e^{-x} -5x^{-2} )dx \\[5mm]
						yx^{-2} & = & -e^{-x} + \dfrac{5}{x} +C, \hspace{1cm} C \in \mathbf{R} \\[5mm]
						y & = & -x^2e^{-x} + 5x+ Cx^2, \hspace{1cm} C \in \mathbf{R}.
					\end{array}
				\]
			\item \((x+xy^2) dx+e^{x^2} ydy=0\). \\[2mm]
				\textbf{Solución.} \fbox{La ecuación es de V.S.} \\[2mm]
				Tenemos que
				\[
					\begin{array}{rcl}
						x(1+y^2) dx +e^{x^2} ydy & = & 0 \\[5mm]
						e^{x^2} ydy & = & -x(1+y^2) dx \\[5mm]
						\dfrac{ydy}{1+y^2} & = & \dfrac{-xdx}{e^{x^2}} \\[5mm]
						\dis\int \dfrac{ydy}{1+y^2} & = & \dis\int \dfrac{-xdx}{e^{x^2}} \\[5mm]
						\dfrac{1}{2} \dis\int \dfrac{du}{1+u} & = & -\dfrac{1}{2} \dis\int \dfrac{dw}{e^{w}} \\[5mm]
					\end{array}
				\]
				\[
					\begin{array}{rcl}
						\log (1+u) & = & +e^{-w} +C \\[5mm]
						\log (1+y^2) & = & e^{-x^2} +C \\[5mm]
						1+y^2 & = & e^{e^{-x^2} +C} \\[5mm]
						y & = & \pm \sqrt{e^{e^{-x^2} +C} -1} \;,\; \hspace{1cm} C \in \mathbf{R}.
					\end{array}
				\]
		\end{enumerate}
	\item \textit{Resuelve la siguiente ecuación diferencial \(x \cdot \dfrac{dy}{dx} = xy \cos x-2xy\), y determina la solución particular explícita que pasa por el punto \((\pi ,1)\).} \\[2mm]
		\textbf{Solución.} Dividamos entre \(x\), todos los términos.
		\[
			\begin{array}{rcl}
				\dfrac{dy}{dx} &=& y \cos x-2y \\[2mm]
				\dfrac{dy}{dx} -y(\cos x-2) & = & 0.
			\end{array}
		\]
		Entonces, se tiene \(p(x) = -\cos x+2\). Luego
		\[
			\mu (x) = e^{\int p} = e^{\int (-\cos x+2) dx} = e^{-\sin x+2x}.
		\]
		Por tanto,
		\[
			\begin{array}{rcl}
				e^{\sin x-2x} \dfrac{dy}{dx} -ye^{- \sin x+2x} (\cos x-2) & = & 0 \\[5mm]
				\dfrac{d}{dy} \big(ye^{- \sin x+2x}\big) & = & 0\\[5mm]
				ye^{- \sin x+2x} & = & k, \hspace{1cm} k \in \mathbf{R} \\[5mm]
				y & = & ke^{\sin x-2x} , \hspace{1cm} k \in \mathbf{R}.
			\end{array}
		\]
		Ahora, se requiere que \(y(\pi) =1\), entonces
		\[
			1 = y(\pi) = ke^{\sin \pi -2 \pi} = ke^{-2 \pi}.
		\]
		Entonces, \(k=e^{2 \pi}\). La solución del P.V.I. \(y(\pi) =1\), es
		\[
			y = e^{2 \pi} \cdot e^{\sin x-2x}.
		\]
\end{enumerate}

\end{document}
