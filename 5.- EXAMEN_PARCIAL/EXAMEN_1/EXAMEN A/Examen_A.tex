\documentclass[10pt,a5paper]{article}

% === PAQUETES === (((
\usepackage[utf8]{inputenc}
\usepackage{amsfonts}
\usepackage{amsmath}
\usepackage{graphicx}
\usepackage{anysize} 
% )))

% === DATOS === (((
% \pdfinfo{%
	% /Title    (Tarea <++> de Ecuaciones Diferenciales)
	% /Author   (Sandra Elizabeth Delgadillo Alemán)
% }
\marginsize{1cm}{1cm}{1cm}{1cm} 
\pagestyle{empty}
% )))

% === TITULO === (((
\newcommand{\titulo}{
	\begin{minipage}{0.25\linewidth}
		\includegraphics[width= 0.9 \linewidth]{IMAGENES/log23.png}
	\end{minipage}
	\begin{minipage}{0.75\linewidth}
		\begin{center}
			\bfseries
			CENTRO DE CIENCIAS BÁSICAS \\
			DEPARTAMENTO DE MATEMÁTICAS Y FÍSICA \\
			ACADEMIA DE MATEMÁTICA AVANZADAS
		\end{center}
	\end{minipage}\\
	\begin{table}[ht]
		\centering
		\begin{tabular}{|*{3}{l|}p{3cm}|}
			\hline
			\textbf{Nombre del Estudiante:} & & \textbf{Fecha:} & \\ \hline
			\textbf{Materia:} & Ecuaciones Diferenciales & \textbf{Carrera:} &  \\ \hline
			\textbf{Profesor:} & Sandra Elizabeth Delgadillo Alemás & \textbf{Semestre:} & \\ \hline
			\textbf{Periodo:} & () Enero--Junio () Agosto--Diciembre & & \\ \hline
			\textbf{Tipo de Examen:} & Parcial: 1() \hspace{2mm} 2() \hspace{2mm} 3() & \textbf{Calificación:} & \\ \hline
		\end{tabular}
	\end{table}
} 
% )))

\begin{document}

\titulo

\begin{enumerate}
	\item \textit{Clasifica las siguientes ecuaciones diferenciales según su tipo,orden y linealidad.}  \\[2mm]
		\textbf{Solución.} 
		\begin{table}[ht]
			\centering
			\begin{tabular}{|*{5} {l|}}
				\hline 
				Ec. & Tipo & Orden & Linealidad & V.S. \\ \hline
				\(\dfrac{d^4x}{dt^4} =x \cdot t\) & Ordinaria & \(4^{to}\) Orden & Lineal & V.S. \\ \hline
				\(y'' -t^2y = \sin t+(y') ^3\) & Ordinaria & \(2^{do}\) Orden & No-lineal & No -V.S. \\ \hline
			\end{tabular}
		\end{table}
	\item \textit{Verifica que \(y(x) = e^{-x} + \dfrac{1}{3} x\) es solución de} 
		\[
			y^{(4)} +4y^{(3)} +3y=x.
		\]
		\textbf{Solución.} Tenemos que
		\[
			\begin{array}{rcl}
				y' & = & -e^{-x} +1/3 \\[2mm]
				y'' & = & e^{-x} \\[2mm]
				y''' & = & -e^{-x} \\[2mm]
				y^{(4)} & = & e^{-x}
			\end{array}
		\]
		Entonces 
		\[
			(e^{-x}) +4(-e^{-x}) +3(e^{-x} +1/3x) =x. \qed
		\]
	\item \textit{Compruebe que \(y+2 \ln y=x^2+1\) es la solución general implícita de la ecuacińo diferencial} 
		\[
			y' = \dfrac{2xy}{y+2}.
		\]
		\textbf{Solución.} Tenemos que
		\[
			\begin{array}{rcl}
				y' +2 \dfrac{y'}{y} =2x & \;\implies\; & y' \Bigg(1+ \dfrac{2}{y}\Bigg) = 2x \\[2mm]
				& \;\implies\; & y' = \dfrac{2x}{1+2/y} = \dfrac{2xy}{y+2}. \qed 
			\end{array}
		\]
	\item \textit{Determina si las siguientes ecuaciones diferenciales son de variables separables o lineales, y resuélvelas.}
		\begin{enumerate}
			\item \(x \cdot \dfrac{dy}{dx} -2y=x^3e^{-x} -3x\). \\[2mm]
				\textbf{Solución.} 
				\fbox{La ecuación diferencial es lineal, no es de V.S.} \\[2mm]
				Primero, como
				\[
					\begin{array}{rcl}
						x \cdot \dfrac{dy}{dx} -2y & = & x^3e^{-x} -3x \\[2mm]
						\dfrac{dy}{dx} - \dfrac{2}{x} \cdot y & = & x^2e^{-x} -3. \\[2mm]
					\end{array}
				\]
				Entonces, \(p(x) = -\dfrac{2}{x}\). Y por tanto, es factor integrante,
				\[
					\mu (x) = e^{\int p} = e^{\int -2/xdx} = e^{-2 \cdot \log x} = x^{-2}.
				\]
				Entonces 
				\[
					\begin{array}{rcl}
						x^{-2} \dfrac{dy}{dx} - \dfrac{2}{x^3} y & = & e^{-x} -3/x^2 \\[5mm]
						\dfrac{d}{dy} \Bigg(x^{-2} \cdot y\Bigg) & = & e^{-x} -3/x^2 \\[5mm]
						x^{-2} \cdot y & = & \dis\int (e^{-x} -3/x^2) dx \\[5mm]
						& = & -e^{-x} +3/x+C.
					\end{array}
				\]
				Así, es que
				\[
					y=-x^2(e^{-x} +3 /x +C), \hspace{5mm} C \in \mathbf{R}. \qed
				\]
			\item \(xy^2dx+e^{x^2} (y^2-1) dy=0\). \\[2mm]
				\textbf{Solución.} \fbox{La ecuación diferencial es de variables separables. No es lineal} \\[2mm]
				Luego, vemos que
				\[
					\begin{array}{rcl}
						e^{x^2} (y^2-1) dy & = & -xy^2dx \\[5mm]
						y^{-2} (y^2-1) dy & = & -e^{-x^2} xdx \\[5mm]
						\dis\int (1-y^{{-2}}) dy & = & \dfrac{1}{2} \dis\int e^{-x^2} (-2x) dx \\[5mm]
						y+ \dfrac{1}{y} & = & e^{-x^2}/2 +C
					\end{array}
				\]
				Así, multiplicando por \(y\) en ambos lados, tenemos que resolver una función cuadrática.
				\[
					y^2-y(e^{-x^2}/2 +C) +1 =0.
				\]
				Así, se tiene que
				\[
					\begin{array}{rcl}
						y & = & \dfrac{+(e^{-x^2}/2 +C) \pm \sqrt{(e^{-x^2}/2 +C) ^2-4(1) (1)}}{2(1)} \\[5mm]
						y & = & \dfrac{e^{-x^2}}{4} + C/2 \pm \sqrt{(e^{-x^2} /2+C) ^2-4}. \hspace{1cm} C \in \mathbf{R}.
					\end{array}
				\]
		\end{enumerate}
	\item \textit{Resuelve la siguiente ecuación diferencial \(x^2 \cdot \dfrac{dy}{dx} = x^2y \cos x -2xy\) y determina la soluciión particular que pasa por el punto \((\pi ,1)\).} \\[2mm]
		\textbf{Solución.} Primero, vemos que
		\[
			\begin{array}{rcl}
				\dfrac{dy}{dx} & = & y \cos x- \dfrac{2y}{x} \\[2mm]
				& = & y \cdot \Bigg(\cos x- \dfrac{2}{x}\Bigg)
			\end{array}
		\]
		Entonces, es de variables separables, y por tanto,
		\[
			\begin{array}{rcl}
				\dfrac{dy}{y} & = & (\cos x-2/x) dx \\[5mm]
				\dis\int \dfrac{dy}{y} & = & \dis\int (\cos x-2/x)dx \\[5mm]
				\log |y| & = & \sin x-2 \log |x|+C \\[5mm]
				|y| & = & e^{\sin x+\log x^{-2}+C} \hspace{1cm} k=e^C>0.\\[5mm]
				y & = & \pm k \cdot e^{\sin x} \cdot e^{\log x^{-2}}, \hspace{1cm} k>0\\[5mm]
				y & = & k \cdot e^{\sin x} \cdot x^{-2}. \hspace{1cm} k \ne 0.
			\end{array}
		\]
		Ahora, se requiere que \(y(\pi) =1\), es decir
		\[
			1 = y(\pi) = k \cdot e^{\sin \pi} \cdot \pi^{-2} = k \cdot \pi^{-2}.
		\]
		Implica que \(k=\pi^2\). La solución del P.V.I. \(y(\pi) =1\), es
		\[
			y(x) = \pi ^2 e^{\sin x} \cdot x^{-2}.
		\]
\end{enumerate}

\end{document}
