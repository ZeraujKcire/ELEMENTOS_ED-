\documentclass[letterpaper,10pt]{memoir}

% === PAQUETES === (((
\usepackage[utf8]{inputenc}
\usepackage{amsfonts}
\usepackage{amsmath}
\usepackage{graphicx}
\usepackage{anysize} 
% )))

% === DATOS === (((
% \pdfinfo{%
	% /Title    (Tarea <++> de Ecuaciones Diferenciales)
	% /Author   (Sandra Elizabeth Delgadillo Alemán)
% }
\marginsize{1cm}{1cm}{1cm}{1cm} 
\pagestyle{empty}
% )))

% === TITULO === (((
\newcommand{\titulo}{
	\begin{minipage}{0.25\linewidth}
		\includegraphics[width= 0.9 \linewidth]{IMAGENES/log23.png}
	\end{minipage}
	\begin{minipage}{0.75\linewidth}
		\begin{center}
			\bfseries
			CENTRO DE CIENCIAS BÁSICAS \\
			DEPARTAMENTO DE MATEMÁTICAS Y FÍSICA \\
			ACADEMIA DE MATEMÁTICA AVANZADAS
		\end{center}
	\end{minipage}\\
	\begin{table}[ht]
		\centering
		\begin{tabular}{|*{3}{l|}p{3cm}|}
			\hline
			\textbf{Nombre del Estudiante:} & & \textbf{Fecha:} & \\ \hline
			\textbf{Materia:} & Ecuaciones Diferenciales & \textbf{Carrera:} &  \\ \hline
			\textbf{Profesor:} & Sandra Elizabeth Delgadillo Alemás & \textbf{Semestre:} & \\ \hline
			\textbf{Periodo:} & () Enero--Junio () Agosto--Diciembre & & \\ \hline
			\textbf{Tipo de Examen:} & Parcial: 1() \hspace{2mm} 2() \hspace{2mm} 3() & \textbf{Calificación:} & \\ \hline
		\end{tabular}
	\end{table}
} 
% )))

\begin{document}

\titulo

\section*{Ley de Enfiramiento de Newton (5).} % (((
\begin{enumerate}
	\item Un termómetro que indica 70°F se coloca en un horno de calentamiento a temperatura constante. A través de una ventana de vidrio del horno un observador registra que la temperatura es igual a 110°F después de medio minuto y de 145°F después de 1 minuto ¿A que temperatura esta el horno?
	\item (Un caballo enfermo) Un veterinario desea saber la temperatura de un caballo enfermo. Las lecturas del termómetro siguen la ley de enfriamiento de Newton. Al momento de insertar el termómetro marca 82°F. Después de 3 minutos la lectura es de 90°F y 3 minutos más tarde de 94°F. Una convulsión repentina destruye el termómetro antes de la lectura final. ¿Cuál es la temperatura del caballo? \textbf{Solución:} 98°F.
	\item En la investigación de un homicidio o una muerte accidental, a menudo es importante estimar el momento de la muerte. A partir de observaciones experimentales se sabe que, con una exactitud satisfactoria en muchas circunstancias, la temperatura superficial de un objeto cambia con una razón proporcional a la diferencia entre la temperatura del objeto y la de su entorno (temperatura ambiente). Esto se conoce como la ley de enfriamiento de Newton. Por tanto, si \(\theta (t)\) en el instante \(t\), suponga que la temperatura del cadáver es de 85°F cuando fue descubierto, y que dos horas más tarde su temperatura es de 74°F, y que la temperatura del medio ambiente es de 68°F.\\
	Determine ¿cuánto tiempo tenía la persona de haber fallecido cuando fue encontrada? (Considere que la temperatura de la persona antes de morir era de 99°F).
	\item Una taza de café caliente, inicialmente a 95°C se enfría hasta 80°C en 5 minutos, al estar en un cuarto con temperatura de 21°C. Use sólo la ley de enfriamiento de Newton y determine el momento en que la temperatura del café estará a unos agradables 50°C. \textbf{Solución:} 20.7 min.
	\item Una cerveza fría, inicialmente a 35°F, se calienta hasta 40°F en 3 minutos, estando en un cuarto con temperatura 70°F. ¿Qué tan caliente estará la cerveza si se deja ahí durante 20 minutos?
	\item Un vino blanco a la temperatura ambiental de 70°F se enfría en hielo (32°F). Si se necesitan 15 minutos para que el vino se enfríe hasta 60°F, ¿cuánto tiempo se necesita para que el vino llegue a los 56°F?. \textbf{Solución:} 22.6 min.
	\item Un vino tinto se saca de la cava, donde estaba a 10°C y se deja respirar en un cuarto con temperatura de 23°C. Si se necesitan 10 minutos para que el vino llegue a los 15°C, ¿en qué momento llegará la temperatura del vino a los 18°C?
	\item Un termómetro se lleva del interior de una habitación al exterior, donde la temperatura del aire es 5°F. Después de un minuto, el termómetro indica 55°F; cinco minutos después marca 30°F. ¿Cuál era la temperatura del interior?
	\item Si una barra metálica pequeña, cuya temperatura inicial es de 20°C, se deja caer en un recipiente con agua hirviente, ¿cuánto tiempo tardará en alcanzar 90°C si se sabe que su temperatura aumentó 2°C en un segundo? ¿Cuánto tiempo tardará en llegar a 98°C?. Solución: 82.1seg, 145.7seg.
	\item Un recipiente con agua hirviendo a 100°C se retira de una estufa en el instante \(t=0\) y se deja enfriar en la cocina. Después de 5 minutos, la temperatura del agua ha descendido a 80°C y otros 5 minutos después ha bajado a 65°C. Suponga que se aplica la ley de enfriamiento de Newton y determine la temperatura de la cocina. Solución: \(T_m=20 ^ \circ C\).
\end{enumerate}
% )))

\section*{Caída Libre (5).} % (((
\textbf{Instrucciones: Hacer 5 ejercicios no resueltos en clase.}\\
A menos que se indique lo contrario, en los siguientes problemas suponemos que la fuerza gravitacional es constante, con \(g = 9.81 m/s ^2\) en el sistema \(MKS \;\;g = 32 pies/s ^2\) en el sistema inglés.
\begin{enumerate}
	\item Un cuerpo con masa de 600g se lanza verticalmente hacia arriba desde el suelo con una velocidad inicial de \(2000cm/s\). Utilice \(g=980cm/s ^2\) y desprecie la resistencia del aire.
		\begin{enumerate}
			\item Calcule el punto más alto y el instante en que llega a ese punto.
			\item Determine la altura del cuerpo y la velocidad después de \(3\) seg. ¿En que momento golpea el cuerpo el suelo?.
		\end{enumerate}
	\item Un objeto de masa \(5 kg\) se libera desde el reposo a \(1000 m\) sobre el suelo y se le permite caer bajo la in fluencia de la gravedad. Suponiendo que la fuerza debida a la resistencia del aire es proporcional a la velocidad del objeto, con constante de proporcionalidad \(b = 50 N \cdot s/m\), determine la ecuación de movimiento del objeto. ¿Cuándo tocará el objeto al suelo?. \textbf{Solución:} 1019 seg.
	\item Si el objeto del problema 2 tiene una masa de 500 kg en vez de 5 kg, ¿cuándo tocará al suelo? [Sugerencia: En este caso, el término exponencial es demasiado grande como para ignorarlo. Use una calculadora o algún software para resolver la ecuación]. \textbf{Solución:} 18.6 seg.
	\item Un objeto de 400 libras se libera desde el reposo a 500 pies sobre el suelo y se le permite caer bajo la influencia de la gravedad. Suponiendo que la fuerza en libras debida a la resistencia del aire es donde es la velocidad del objeto en \(pies/s\), determine la ecuación de movimiento del objeto. ¿En qué momento tocará el objeto al suelo?
	\item Una persona deja caer una piedra desde un edificio y espera 1.5 seg.; luego lanza una pelota de béisbol con una velocidad inicial de \(20m/s\). Desprecie la resistencia del aire. Si la piedra y la pelota chocan contra el suelo al mismo tiempo, ¿Cuál es la altura del Edificio?
	\item Un objeto de masa 5 kg recibe una velocidad inicial hacia abajo de \(50 m/s\) y luego se le permite caer bajo la influencia de la gravedad. Suponga que la fuerza en newtons debida a la resistencia del aire es \(-10v\), donde \(v\) es la velocidad del objeto en \(m/s\). Determine la ecuación de movimiento del objeto. Si el objeto está inicialmente a 500 m sobre el suelo, determine el momento en que el objeto golpeará el suelo. \textbf{Solución:} 97.3 seg.
	\item Un objeto de masa 8 kg recibe una velocidad inicial hacia arriba de \(20 m/s\) y luego se le permite caer bajo la influencia de la gravedad. Suponga que la fuerza en newtons debida a la resistencia del aire es \(-16v\), donde \(v\) es la velocidad del objeto en \(m/s\). Determine la ecuación de movimiento del objeto. Si el objeto está en un principio a 100m sobre el suelo, determine el momento en que el objeto golpeará el suelo.
	\item Una paracaidista cuya masa es de 75 kg se arroja de un helicóptero que vuela a 2000 m sobre el suelo y cae hacia éste bajo la influencia de la gravedad. Suponga que la fuerza debida a la resistencia del aire es proporcional a la velocidad de la paracaidista, con la constante de proporcionalidad \(b = 30 N \cdot s/m\) cuando el paracaídas está cerrado y \(b = 90 N \cdot s/m\) cuando se abre. Si el paracaídas no se abre hasta que la velocidad de la paracaidista es de \(20 m/s\). ¿después de cuántos segundos llegará ella al suelo? \textbf{Solución:} 242 seg.
	\item Cuando un cuerpo como el de un paracaidista que antes de que se abra el paracaídas se mueve a gran rapidez en el aire, la resistencia del mismo se describe mejor con la velocidad instantánea elevada a cierta potencia. Formule una ecuación diferencial que relacione la velocidad \(v(t)\) de un cuerpo de masa ni que cae, si la resistencia del aire es proporcional al cuadrado de la velocidad instantánea.
	\item Considere una pelota perforada, ¿qué toma más tiempo, subir o bajar?. Considere que la resistencia del aire es proporcional a la velocidad de la pelota.
\end{enumerate}
% )))

\end{document}
