\documentclass[letterpaper,10pt]{memoir}

% === PAQUETES === (((
\usepackage[utf8]{inputenc}
\usepackage{amsfonts}
\usepackage{amsmath}
\usepackage{graphicx}
\usepackage{anysize} 
% )))

% === DATOS === (((
% \pdfinfo{%
	% /Title    (Tarea <++> de Ecuaciones Diferenciales)
	% /Author   (Sandra Elizabeth Delgadillo Alemán)
% }
\marginsize{1cm}{1cm}{1cm}{1cm} 
\pagestyle{empty}
% )))

% === TITULO === (((
\newcommand{\titulo}{
	\begin{minipage}{0.25\linewidth}
		\includegraphics[width= 0.9 \linewidth]{IMAGENES/log23.png}
	\end{minipage}
	\begin{minipage}{0.75\linewidth}
		\begin{center}
			\bfseries
			CENTRO DE CIENCIAS BÁSICAS \\
			DEPARTAMENTO DE MATEMÁTICAS Y FÍSICA \\
			ACADEMIA DE MATEMÁTICA AVANZADAS
		\end{center}
	\end{minipage}\\
	\begin{table}[ht]
		\centering
		\begin{tabular}{|*{3}{l|}p{3cm}|}
			\hline
			\textbf{Nombre del Estudiante:} & & \textbf{Fecha:} & \\ \hline
			\textbf{Materia:} & Ecuaciones Diferenciales & \textbf{Carrera:} &  \\ \hline
			\textbf{Profesor:} & Sandra Elizabeth Delgadillo Alemás & \textbf{Semestre:} & \\ \hline
			\textbf{Periodo:} & () Enero--Junio () Agosto--Diciembre & & \\ \hline
			\textbf{Tipo de Examen:} & Parcial: 1() \hspace{2mm} 2() \hspace{2mm} 3() & \textbf{Calificación:} & \\ \hline
		\end{tabular}
	\end{table}
} 
% )))

\begin{document}

\titulo

\begin{enumerate}
	\item Determine si las siguientes ecuaciones son de variables y/o lineales:
		\begin{center}
			\begin{tabular}{|p{7cm}|p{7cm}|}
				\hline 
				\begin{enumerate}
					\item \(x^2 \dfrac{dy}{dx} + \cos x=y\).
					\item \((y-4x-1) ^2dx-dy =0\).
					\item \((t+x+2) dx+(3t-x-6)dt =0\).
				\end{enumerate}
				&
				\begin{enumerate}
					\setcounter{enumii}{3}
					\item \((y^3e^{-2x} +y^3) dx-e^{-2x} dy=0\).
					\item \(\dfrac{1}{y-3} \cdot \dfrac{dy}{dx} =- \dfrac{1}{2}\).
					\item \((t^2-1) \dfrac{dy}{dt} =yt-y\).
				\end{enumerate}
				\\ \hline
			\end{tabular}
		\end{center} 
\end{enumerate}

\section*{Ecuaciones Diferenciales de Variables Separables (10).} % (((
\begin{enumerate}
	\item \textbf{(Hacer 7 ejercicios)} Resuelve las siguientes ecuaciones diferenciales, usando el método de separación de variables y encuentra las soluciones explícitas cuando sea posible.\\[2mm]
		\begin{minipage}{0.5\linewidth}
			\begin{enumerate}
				\item \(\dfrac{dy}{dx} = \sin 5x\). \textbf{Solución.} \(y=- \dfrac{1}{5} \cos 5x+c\).
				\item \(dx+e^{3x} dy=0\). \textbf{Solución.} \(y= \dfrac{ 1}{3} e^{-3x} +c\).
				\item \(x \dfrac{dy}{dx} =4y\). \textbf{Solución.} \(y=cx^4\).
				\item \(\dfrac{dy}{dx} =e^{3x+2y}\). \textbf{Solución.} \(-3e^{-2y} =2e^{3x} +c\).
				\item \(y \ln x \cdot \dfrac{dy}{dx} = \Bigg(\dfrac{y+1}{x}\Bigg) ^2\). \textbf{Solución.} \(\dfrac{1}{3} x^3 \ln x- \dfrac{1}{9} x^3 = \dfrac{1}{2} y^2+2y+ \ln |y|+c\).
				\item \(\sec ^2x dy+ \csc ydx=0\). \textbf{Solución.} \(4 \cos y=2x+ \sin 2x+c\).
			\end{enumerate}
		\end{minipage}\hspace{5mm}
		\begin{minipage}{0.5\linewidth}
			\begin{enumerate}
			\setcounter{enumii}{6}
			\item \((e^y+1) ^2e^{-y} dx+(e^x+1) ^3e^{-x} dy=0\). \textbf{Solución.} \((e^x+1) ^{-2} +2(e^y+1) ^{-1} =c\).
			\item \(\dfrac{dS}{dr} =kS\). \textbf{Solución.} \(S=ce^{kr}\).
			\item \(\dfrac{dP}{dt} =P-P^2\). \textbf{Solución.} \(P= \dfrac{ce^t}{1+ce^t}\).
			\item \(\dfrac{dy}{dx} = \dfrac{xy+3x-y-3}{xy-2x+4y-8}\). \textbf{Solución.} \((y+3) ^5e^x=c(x+4) ^5e^y\).
			\item \(\dfrac{dy}{dx} =x \sqrt{1-y^2}\). \textbf{Solución.} \(y= \sin \Bigg(\dfrac{1}{2} x^2+c\Bigg)\).
			\end{enumerate}
		\end{minipage}
	\item \textbf{(Hacer 2 ejercicios)} Resuelve los siguientes problemas de valor inicial.
		\begin{enumerate}
			\item \(\dfrac{dx}{dt}= 4(x^2+1)\), \(x \Bigg(\dfrac{\pi}{4}\Bigg) =1\). \textbf{Solución.} \(x= \tan \Bigg(4t- \dfrac{3 \pi}{4}\Bigg)\).
			\item \(y \,' =xy^3(1+x^2) ^{-1/2}\), \(y(0) =1\).
		\end{enumerate}
	\item \textbf{(Hacer este ejercicio)} Demuestre una solución implícita de
		\[
			2x \sin ^2ydx-(x^2+10) \cos ydy=0,
		\]
		es \(\ln (x^2+10) + \csc y=c\), resolviendo la E.D. y derivando implícitamente.
\end{enumerate}
% )))

\section*{Ecuaciones Diferenciales Lineales (10).} % (((
\begin{enumerate}
	\item \textbf{(Resuelve 7 de estos problemas)} En los siguientes problemas determine la solución general de la ecuación diferencial correspondiente. Indica cuál es el mayor intervalo en el cual esté definida la solución general.
		\begin{enumerate}
			\item \(\dfrac{dy}{dx} =5y\). \textbf{Solución.} \(y=ce^{5x}\), \(x \in \mathbb{R}\).
			\item \(\dfrac{dy}{dx} +y=e^{3x}\). \textbf{Solución.} \(y= \dfrac{1}{4} e^{3x} +ce^{-x}\), \(x \in \mathbb{R}\).
			\item \(y \,' +3x^2y=x^2\). \textbf{Solución.} \(y= \dfrac{1}{3} +ce^{-x^3}\), \(x \in \mathbb{R}\).
			\item \(x \dfrac{dy}{dx} -y=x^2 \sin x\). \textbf{Solución.} \(y=cx-x \cos x\), \(x \in \mathbb{R}\).
			\item \(x \dfrac{dy}{dx} +4y=x^3-x\). \textbf{Solución.} \(y= \dfrac{1}{7} x^3- \dfrac{1}{5} x+cx^{-4}\), \(x>0\).
			\item \(x^2y \,' +x(x+2) y=e^x\). \textbf{Solución.} \(y= \dfrac{1}{2x^2} e^x+ \dfrac{c}{x^2} e^{-x}\), \(x>0\).
			\item \(ydx-4(x+y^6) dy=0\). \textbf{Solución.} \(x=2y^6+cy^4\), \(y>0\).
			\item \(\cos x \cdot \dfrac{dy}{dx} + (\sin x) y=1\). \textbf{Solución.} \(y= \sin x+c \cdot \cos x\), \(x \in (- \pi /2 \;,\; \pi /2)\).
			\item \((t+y+1) dt-dy=0\). \textbf{Solución.} \(y(t) =-t-2+ce^t\).
			\item \(x \dfrac{dy}{dx} +(3x+1) y=e^{-3x}\). \textbf{Solución.} \(y=e^{-3x} + \dfrac{c}{x} e^{-3x}\), \(x>0\).
		\end{enumerate}
	\item \textbf{(Resuelve 2 de estos ejercicios)} En los siguientes problemas resuelve el problema de valor inicial respectivo. Describa el mayor intervalo en el cual esté definida la solución.
		\begin{enumerate}
			\item \(xy \,' +y=e^x\), \(y(1) =2\). \textbf{Solución.} \(y= \dfrac{e^x}{x} + \dfrac{2-e}{x}\), \(x>0\).
			\item \(t^3 \dfrac{dx}{dt} +3t^2x=t\), \(x(2) =0\). \textbf{Solución.} \(x(t) = \dfrac{1}{2} t^{-1} -2t^{-3}\).
			\item \(xy \,' +2y=x^2-x+1\), \(y(1) =1/2\), \(x>0\).
			\item \(y \,' +y= \dfrac{1}{1+x^2}\), \(y(0) =0\).
		\end{enumerate}
	\item \textbf{(Hacer este ejercicio)} Encuentre la solución del siguiente problema de valor inicial \(\dfrac{dy}{dx} = \dfrac{1}{e^y-x}\), \(y(1) =0\). Sugerencia: Considere a \(x\) como la variable dependiente, en vez de \(y\).
\end{enumerate}
% )))

\end{document}
