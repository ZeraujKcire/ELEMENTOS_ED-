\documentclass[letterpaper,10pt]{memoir}


% === PAQUETES === (((
\usepackage[utf8]{inputenc}
\usepackage{amsfonts}
\usepackage{amsmath}
\usepackage{graphicx}
\usepackage{anysize} 
% )))

% === DATOS === (((
% \pdfinfo{%
	% /Title    (Tarea <++> de Ecuaciones Diferenciales)
	% /Author   (Sandra Elizabeth Delgadillo Alemán)
% }
\marginsize{1cm}{1cm}{1cm}{1cm} 
\pagestyle{empty}
% )))

% === TITULO === (((
\newcommand{\titulo}{
	\begin{minipage}{0.25\linewidth}
		\includegraphics[width= 0.9 \linewidth]{IMAGENES/log23.png}
	\end{minipage}
	\begin{minipage}{0.75\linewidth}
		\begin{center}
			\bfseries
			CENTRO DE CIENCIAS BÁSICAS \\
			DEPARTAMENTO DE MATEMÁTICAS Y FÍSICA \\
			ACADEMIA DE MATEMÁTICA AVANZADAS
		\end{center}
	\end{minipage}\\
	\begin{table}[ht]
		\centering
		\begin{tabular}{|*{3}{l|}p{3cm}|}
			\hline
			\textbf{Nombre del Estudiante:} & & \textbf{Fecha:} & \\ \hline
			\textbf{Materia:} & Ecuaciones Diferenciales & \textbf{Carrera:} &  \\ \hline
			\textbf{Profesor:} & Sandra Elizabeth Delgadillo Alemás & \textbf{Semestre:} & \\ \hline
			\textbf{Periodo:} & () Enero--Junio () Agosto--Diciembre & & \\ \hline
			\textbf{Tipo de Examen:} & Parcial: 1() \hspace{2mm} 2() \hspace{2mm} 3() & \textbf{Calificación:} & \\ \hline
		\end{tabular}
	\end{table}
} 
% )))


\begin{document}

\titulo

\begin{enumerate}
	\item En los siguientes problemas se proporciona la solución uniparamétrica de las E.D. Encuentre la solución de los P.V.I. formados por la E.D. y la condición inicial que se proporciona. De el intervalo de definición más grande en el cual se define la solución y esboza la gráfica (si es necesario usa una Gogebra o WolframAlpha).
		\begin{enumerate}
			\item P.V.I. \(y \,' =y-y^3\) con \(y(0) =- \dfrac{1}{3}\). \textbf{Solución.} Sol. Gral. \(y(x) = \dfrac{1}{1+ce^{-x}}\).
			\item P.V.I. \(y \,' +2xy^2=0\) con \(y(-2) = \dfrac{1}{2}\). \textbf{Solución.} Sol. Gral. \(y(x) = \dfrac{1}{x^2+c}\).
		\end{enumerate}
	\item Sea \(y(x) =2x+ce^{-x}\) de la ecuación diferencial \(y \,' +y=2+2x\).
		\begin{enumerate}
			\item Determina la solución particular que pasa por el punto \((x_0,y_0)\).
			\item Considera el punto \((1,1)\) e indica que función solución pasa por dicho punto.
			\item Esboza la familia de soluciones de la E.D. dada e identifica la curva de la función solución encontrada en b).
		\end{enumerate}
	\item Resuelve el problema de valor inicial \(ty'+2y= \sin  t\), \(y(\pi /2)=1\), \(t>0\).  Utilice WolframAlpha para determinar la solución general de la ecuación diferencial.
	\item Compruebe que \(3x^2+y^2=c\) es la solución general implícita de la E.D. \(\dfrac{dy}{dx} =- \dfrac{3x}{y}\).
		\begin{enumerate}
			\item Encuentre la solución particular implícita que satisface el P.V.I conformado por la E.D. dada y la condición inicial \(y(-2) =3\).
			\item Encuentre las soluciones explícitas  de la E.D. \(y=y(x)\) e indica cual es su intervalo de definición.
			\item Grafíca algunas soluciones explícitas de la ecuación diferencial e indica cual gráfica corresponde a la solución del P.V.I.
		\end{enumerate}
	\item Aproveche que se da la solución general de las ecuaciones diferenciales, para determinar una solución de los problemas de valores iniciales formados por la ecuación y las condiciones iniciales indicadas. Esboza la gráfica de la solución de los P.V.I. para mostrar la interpretación geométrica de las condiciones iniciales.
		\begin{enumerate}
			\item \(x(t) =c_1 \cos t+c_2 \sin t\), \(x '' +x=0\).
				\begin{enumerate}
					\item \(x(0) =-1\), \(x \,' (0) =8\).
					\item \(x \Big(\dfrac \pi 6\Big) = \dfrac{1}{2}\), \(x \,' \Big(\dfrac{\pi}{6}\Big) =0\).
				\end{enumerate}
			\item \(y(x) =c_1e^x+c_2e^{3x} + \dfrac{1}{3}\), \(y '' -4y \,' +3y=1\).
				\begin{enumerate}
					\item \(y(0) =y \,' (0) =1\).
					\item \(y(-1) =0\), \(y \,' (-1) =-5\).
				\end{enumerate}
		\end{enumerate}
\end{enumerate}

{\large \textbf{Problemas para Discusión.}}
\begin{enumerate}
	\item Encuentra una función \(y=y(x)\) cuya gráfica en cada punto \((x,y)\) tiene la pendiente dada por \(8e^{2x} +6x\) y la ordenada al origen \((0,9)\).
	\item En la Figura muestra las gráficas de cuatro miembros de una familia de soluciones de la ecuación diferencial de segundo orden \(\dfrac{d^2y}{dx^2} =f(x,y,y \,')\). Determine la correspondencia de cada curva solución con las condiciones iniciales adecuadas.\\
		\begin{minipage}{0.4\linewidth}
			\begin{enumerate}
				\item \(y(1) = 1, y'(1) = -2\).
				\item \(y(-1) = 1, y'(-1) = -4\).
				\item \(y(1) = 1, y'(1) = 2\).
				\item \(y(0) = -1, y' (0) = 2\).
				\item \(y(0) = -1, y'(0) = 0\).
				\item \(y(0) = -4, y' (0) = -2\).
			\end{enumerate}
		\end{minipage}\hspace{5mm}
		\begin{minipage}{0.6\linewidth}
				\includegraphics[width= 0.9 \linewidth]{IMAGENES/images/image5.png}
		\end{minipage}
	\item Considere el problema de valores iniciales \(y \,' =x-2y\), \(y(0) = \frac 12\). Determine cuál de las dos curvas que se muestran en la siguiente figura es la única curva solución plausible.\\Explique su razonamiento.
		\begin{figure}[ht]
			\centering
			\includegraphics[width= 0.5 \linewidth]{IMAGENES/images/image10.png}
		\end{figure}
\end{enumerate}

\end{document}
