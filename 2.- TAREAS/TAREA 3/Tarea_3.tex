\documentclass[letterpaper,10pt]{memoir}

% === PAQUETES === (((
\usepackage[utf8]{inputenc}
\usepackage{amsfonts}
\usepackage{amsmath}
\usepackage{graphicx}
\usepackage{anysize} 
% )))

% === DATOS === (((
% \pdfinfo{%
	% /Title    (Tarea <++> de Ecuaciones Diferenciales)
	% /Author   (Sandra Elizabeth Delgadillo Alemán)
% }
\marginsize{1cm}{1cm}{1cm}{1cm} 
\pagestyle{empty}
% )))

% === TITULO === (((
\newcommand{\titulo}{
	\begin{minipage}{0.25\linewidth}
		\includegraphics[width= 0.9 \linewidth]{IMAGENES/log23.png}
	\end{minipage}
	\begin{minipage}{0.75\linewidth}
		\begin{center}
			\bfseries
			CENTRO DE CIENCIAS BÁSICAS \\
			DEPARTAMENTO DE MATEMÁTICAS Y FÍSICA \\
			ACADEMIA DE MATEMÁTICA AVANZADAS
		\end{center}
	\end{minipage}\\
	\begin{table}[ht]
		\centering
		\begin{tabular}{|*{3}{l|}p{3cm}|}
			\hline
			\textbf{Nombre del Estudiante:} & & \textbf{Fecha:} & \\ \hline
			\textbf{Materia:} & Ecuaciones Diferenciales & \textbf{Carrera:} &  \\ \hline
			\textbf{Profesor:} & Sandra Elizabeth Delgadillo Alemás & \textbf{Semestre:} & \\ \hline
			\textbf{Periodo:} & () Enero--Junio () Agosto--Diciembre & & \\ \hline
			\textbf{Tipo de Examen:} & Parcial: 1() \hspace{2mm} 2() \hspace{2mm} 3() & \textbf{Calificación:} & \\ \hline
		\end{tabular}
	\end{table}
} 
% )))

\begin{document}
	
\titulo

\begin{enumerate}
	\item Use \textbf{dfield} para dibujar el campo direccional respectivo, y traza a mano las curvas solución que pase por los puntos indicados.
		\begin{enumerate}
			\item \(\dfrac{dy}{dx} = \dfrac{x}{y}\), \(y(0) =5\), \(y(3) =3\), \(y(4) =2\), \(y(-5) =-3\).
			\item \(\dfrac{dy}{dx} =1-xy\), \(y(0) =0\), \(y(-1) =0\), \(y(2) =2\), \(y(0) =-4\).
		\end{enumerate}
	\item Use el método de las isóclinas para dibujar el campo direccional de 3 de las siguientes E.D. y esboza algunas soluciones de la E.D. incluyendo las soluciones de que satisfacen las condiciones dadas.\\[2mm]
		\begin{minipage}{0.5\linewidth}
			\begin{enumerate}
				\item \(\dfrac{dy}{dx}+y=3+ \dfrac{1}{x}\), \(y(1) =0\).
				\item \(\dfrac{dy}{dx} -y=0.2x^3\), \(y(0) = \dfrac{1}{2}\), \(y(2) =-1\).
				\item \(y \,' =y- \cos \dfrac{\pi}{2} x\), \(y(2) =2\), \(y(-1) =0\).
			\end{enumerate}
		\end{minipage}\hspace{5mm}
		\begin{minipage}{0.5\linewidth}
			\begin{enumerate}
			\setcounter{enumii}{3}
			\item \(\dfrac{dy}{dx} = \dfrac{1}{y}\), \(y(0) =1\), \(y(-2) =-1\).
			\item \(\dfrac{dy}{dx} =x^2+2y^2\), \(y(0) =1\).
			\item \(y \,' =y- \sin (\sin x)\), \(y(0) =0\).
			\end{enumerate}
		\end{minipage}
	\item Cada figura representa la gráfica de \(f(y)\) y \(f(x)\) respectivamente. A mano trace un campo de direcciones y algunas soluciones en una rejilla adecuada para \(y \,' =f(y)\), y \(y \,' =f(x)\). \\[2mm]
		\begin{minipage}{0.5\linewidth}
			(a)\\
			\includegraphics[width= 0.8 \linewidth]{IMAGENES/images/image19.png}
		\end{minipage}\hspace{5mm}
		\begin{minipage}{0.5\linewidth}
			(b)\\
			\includegraphics[width= 0.8 \linewidth]{IMAGENES/images/image21.png}
		\end{minipage}
	\item Elige 3 de las siguientes ecuaciones diferenciales autónomas. a) Determina sus puntos de equilibrio y clasifícalos usando la línea fase. b) Esboza algunas curvas solución con base a la información que nos da la tabla de características de monotonía y concavidad. Además, resalta la solución que satisface la condición inicial \(y(0) =1\).
		\begin{enumerate}
			\item \(y \,' -y^2+1=0\).
			\item \(\dfrac{dy}{dx} =10+3y-y^2\).
			\item \(y \,' =y^2(4-y^2)\).
			\item \(\dfrac{dy}{dx} - \cos y=0\).
			\item \(y \,' =y(y^2-2y-8)\).
		\end{enumerate}
	\item Para 2 las siguientes ecuaciones diferenciales, utilice la línea fase para predecir el comportamiento asintótico cuando \(t \longrightarrow \infty\) para la solución que satisface la condición inicial dada.
		\begin{enumerate}
			\item \(y \,' =2y^2(1-y^2)\), \(y(0) =0.5\).
			\item \(y \,' = y \ln (y+2)\), \(y(0) =1\).
			\item \(y \,' =y-y^3\), \(y(0) =1.1\).
		\end{enumerate}
\end{enumerate}
\textbf{NOTA:} : Revisa tu tarea con el software libre \texttt{dfield.jar} (\url{http://math.rice.edu/~dfield/dfpp.html})

\end{document}
