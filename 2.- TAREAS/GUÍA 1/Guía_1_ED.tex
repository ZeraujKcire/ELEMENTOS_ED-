\documentclass[letterpaper,10pt]{memoir}

% === PAQUETES === (((
\usepackage[utf8]{inputenc}
\usepackage{amsfonts}
\usepackage{amsmath}
\usepackage{graphicx}
\usepackage{anysize} 
% )))

% === DATOS === (((
% \pdfinfo{%
	% /Title    (Tarea <++> de Ecuaciones Diferenciales)
	% /Author   (Sandra Elizabeth Delgadillo Alemán)
% }
\marginsize{1cm}{1cm}{1cm}{1cm} 
\pagestyle{empty}
% )))

% === TITULO === (((
\newcommand{\titulo}{
	\begin{minipage}{0.25\linewidth}
		\includegraphics[width= 0.9 \linewidth]{IMAGENES/log23.png}
	\end{minipage}
	\begin{minipage}{0.75\linewidth}
		\begin{center}
			\bfseries
			CENTRO DE CIENCIAS BÁSICAS \\
			DEPARTAMENTO DE MATEMÁTICAS Y FÍSICA \\
			ACADEMIA DE MATEMÁTICA AVANZADAS
		\end{center}
	\end{minipage}\\
	\begin{table}[ht]
		\centering
		\begin{tabular}{|*{3}{l|}p{3cm}|}
			\hline
			\textbf{Nombre del Estudiante:} & & \textbf{Fecha:} & \\ \hline
			\textbf{Materia:} & Ecuaciones Diferenciales & \textbf{Carrera:} &  \\ \hline
			\textbf{Profesor:} & Sandra Elizabeth Delgadillo Alemás & \textbf{Semestre:} & \\ \hline
			\textbf{Periodo:} & () Enero--Junio () Agosto--Diciembre & & \\ \hline
			\textbf{Tipo de Examen:} & Parcial: 1() \hspace{2mm} 2() \hspace{2mm} 3() & \textbf{Calificación:} & \\ \hline
		\end{tabular}
	\end{table}
} 
% )))

\begin{document}

\titulo

\section{Ejercicios de repaso.} % (((
\begin{enumerate}
	\item Obtener la \(1^{er}\) y \(2^{da}\) derivada de las siguientes funciones:\\[3mm]
		\begin{minipage}{0.3\linewidth}
			\begin{enumerate}
				\item \(g(s) = 3s^2-2s^4\).
				\item \(h(x) = 6 \sqrt{x} +3 \sqrt[3]{x}\).
				\item \(f(x) = 4x-5 \sin x\).
				\item \(h(t) = \dfrac{8}{5t^4}\).
				\item \(y(x) = \dfrac{x^4}{\cos x}\).
			\end{enumerate}
		\end{minipage}\hspace{5mm}
		\begin{minipage}{0.3\linewidth}
			\begin{enumerate}
				\setcounter{enumii}{5}
			\item \(f(x) = (2x^3+5x) (3x-4)\).
			\item \(y(x) = x \cos x- \sin x\).
			\item \(y(x)=(7x +3) ^4\).
			\item \(f(x) = \dfrac{1}{(5x+1) ^2}\).
			\item \(h(x) = 5 \cos (9x+1)\).
			\end{enumerate}
		\end{minipage}
		\begin{minipage}{0.3\linewidth}
			\begin{enumerate}
				\setcounter{enumii}{10}
				\item \(f(x) = \sqrt{1-x^3}\).
				\item \(f(x) = \dfrac{3x}{\sqrt{x^2+1}}\).
				\item \(h(x) = \bigg(\dfrac{x+5}{x^2+3}\bigg) ^2\).
			\end{enumerate}
		\end{minipage} 
	\item Utilice las tablas de integración para determinar las integrales indefinidas dadas.\\[2mm]
		\begin{minipage}{0.3\linewidth}
			\begin{enumerate}
				\item \(\dis\int (x-6) dx\).
				\item \(\dis\int \dfrac{6}{\sqrt[3]{x}} dx\).
				\item \(\dis\int (2x-9 \sin x) dx\).
				\item \(\dis\int \dfrac{x^2}{\sqrt{x^3+3}} dx\).
				\item \(\dis\int x \sin 3x^2 dx\).
			\end{enumerate}
		\end{minipage} 
		\begin{minipage}{0.3\linewidth}
			\begin{enumerate}
				\setcounter{enumii}{5}
				\item \(\dis\int \dfrac{\cos \theta}{\sqrt{1- \sin \theta}} d \theta\).
				\item \(\dis\int 2xe^{2x} dx\).
				\item \(\dis\int \dfrac{1}{7x-2} dx\).
				\item \(\dis\int (x+1) e^{(x+1) ^2} dx\).
				\item \(\dis\int \dfrac{\sin x}{1+ \cos x} dx\).
			\end{enumerate}
		\end{minipage} 
		\begin{minipage}{0.3\linewidth}
			\begin{enumerate}
				\setcounter{enumii}{10}
				\item \(\dis\int \dfrac{e^{4s} -e^{2s} +1}{x^2e^{x^3} +1} ds\).
				\item \(\dis\int x^2e^{x^3+1} ds\).
				\item \(\dis\int \dfrac{x}{9-x^4} dx\).
			\end{enumerate}
		\end{minipage} 
	\item Clasifica las siguientes E.D.\\[2mm]
		\begin{minipage}{0.3\linewidth}
			\begin{enumerate}
				\item \(\dot{y} = \sin t-t^3 y\).
				\item \(y'' -t^2y= \sin t+(y') ^2\).
				\item \(e^ty'' +(\sin t) y' +3y=5e^t\).
			\end{enumerate}
		\end{minipage} 
		\begin{minipage}{0.3\linewidth}
			\begin{enumerate}
				\setcounter{enumii}{3}
				\item \(\bigg(\dfrac{d^2y}{dx^2}\bigg) ^2 = y^5\).
				\item \((t^2+y^2) ^{1/2} =y' +t\).
				\item \(y' = (1+t^2) y'' - \cos t\).
			\end{enumerate}
		\end{minipage} 
		\begin{minipage}{0.3\linewidth}
			\begin{enumerate}
				\setcounter{enumii}{6}
				\item \(\dfrac{dy}{dt} = \dfrac{1}{t^2-t}\).
				\item \(5 \dfrac{d^2x}{dt^2} +4 \dfrac{d^2y}{dx^2} +9x = 2 \cos 3x\).
				\item \(8 \dfrac{dy}{dx^2} = x(1-x)\).
			\end{enumerate}
		\end{minipage} 
	\item Resuelve las E.D. de V.S. \\[2mm]
		\begin{minipage}{0.5\linewidth}
		\begin{enumerate}
			\item \(\dfrac{dx}{dt} = 3xt^2\).
			\item \(\dfrac{dy}{dx} = \dfrac{x}{y^2 \sqrt{1+x}}\).
			\item \(\dfrac{dx}{dt} = \dfrac{t}{xe^{t+2x}}\).
			\item \(\dfrac{dy}{dx} = \dfrac{\sec ^2y}{1+x^2}\).
		\end{enumerate}
		\end{minipage} 
		\begin{minipage}{0.5\linewidth}
		\begin{enumerate}
			\setcounter{enumii}{4}
			\item \(\dfrac{dy}{dx} = 3x^2(1+y^2) ^{3/3}\).
			\item \(y^{-1} dy +ye^{\cos x} \sin xdx=0\).
			\item \(\dfrac{1}{2} \dfrac{dy}{dx} = \sqrt{y+1} \cos x\), \(y(\pi) =0\).
			\item \(\sqrt{y} dx + (1+x) dy=0\) , \(y(0) =1\).
			\item \(\dfrac{1}{\theta} \dfrac{dy}{d \theta} = \dfrac{y \sin \theta}{y^2+1}\).
		\end{enumerate}
		\end{minipage} 
	\item Resuelve las siguientes E.D. lineales. \\[2mm]
		\begin{minipage}{0.5\linewidth}
			\begin{enumerate}
				\item \(\dfrac{dy}{dx} = \dfrac{y}{x} +2x+1\).
				\item \(\dfrac{dy}{dx} -t-e^{3x} =0\).
				\item \(t(2y-1) +2y' =0\).
				\item \(y \dfrac{dx}{dy} +2x=5y^3\).
			\end{enumerate}
		\end{minipage} 
		\begin{minipage}{0.5\linewidth}
			\begin{enumerate}
				\setcounter{enumii}{4}
				\item \(y' = \sin t-y \sin t\).
				\item \((x^2+1) \dfrac{dy}{dx} +xy-x=0\).
				\item \((3t-y) +2ty' =0\).
				\item \(y' +(\cos t) y=2 \cos t\).
				\item \(\dfrac{dy}{dx} + \dfrac{3}{x} y +2 = 3x\), \(y(1) =1\).
			\end{enumerate}
		\end{minipage} 
	\item Verifica si las E.D. son separables, lineales, ambas, o ninguna. \\[2mm]
		\begin{minipage}{0.5\linewidth}
			\begin{enumerate}
				\item \(\dfrac{dx}{dt} +xt=e^x\).
				\item \(3t=e^t \dfrac{dy}{dt} +y \ln x\).
				\item \(3r= \dfrac{dr}{d \theta} - \theta ^3\).
			\end{enumerate}
		\end{minipage}\hspace{5mm}
		\begin{minipage}{0.5\linewidth}
			\begin{enumerate}
				\setcounter{enumii}{3}
			\item \(x^2 \dfrac{dy}{dx} + \sin x-y=0\).
			\item \((t^2+1) \dfrac{dy}{dt} =yt-y\).
			\item \(x \dfrac{dx}{dt} +t^2x= \sin t\).
			\end{enumerate}
		\end{minipage}
\end{enumerate}
% )))

\section{Ejercicios de Razonamiento.} % (((
\begin{enumerate}
	\item Muestra que \(x^2+x-3=0\) es una solución implícita de \(\dfrac{dy}{dx} =- \dfrac{1}{2y}\), en el intervalo \((- \infty ,3)\).
	\item Muestr que \(xy^3-xy^3 \sin x=1\) es una solución implícita de
		\[
			\dfrac{dy}{dx} = \dfrac{(\cos x+ \sin x-1) y}{3(x-x \sin x)}.
		\]
	\item Muestra que \(\phi (x) =e^x-x\) es una solución explícita de
		\[
			\dfrac{dy}{dx} +y^2=e^{2x} +(1-2x) e^x+x^2-1.
		\]
	\item Muestra que \(\phi (x) =x^2-x^{-1}\) es una solución explícita de \(x^2 \dfrac{d^2y}{dx^2} =2y\), en el intervalo \((0, \infty)\).
	\item Determina, si \(e^{xy} +y=x-1\) es solución implícita de la E.D. \(\dfrac{dy}{dx} = \dfrac{e^{-xy} -y}{e^{-xy} +x}\). Asuma que \(y=y(x)\).
	\item Verifica que \(\phi (x) = c_1 \sin x+c_2 \cos x\) es una solución de \(\dfrac{d^2y}{dx^2} +y=0\), para cualquier constantes \(c_1,c_2\). Entonces \(c_1 \sin x+ c_2 \cos x\) es una familia de soluciones de la E.D. con dos parámetros.
	\item Verifica que \(x^2+cy^2=1\), donde \(c\) es una constante abritraria diferente de cero, es una familia de un sólo parámetro, de soluciones de
		\[
			\dfrac{dy}{dx} = \dfrac{xy}{x^2-1}.
		\]
	\item Muestra que \(\phi (x) =Ce^{3x} +1\) es una solución de \(\dfrac{dy}{dx} -3y=-3\) para cualquier constante \(C\). Entonces \(Ce^x+1\) es una familia de soluciones de la E.D. con un sólo parámetro. Grafica algunas funciones solución.
	\item Sea \(c>0\). Muestra que la función \(\phi (x) =(c^2-x^2) ^{-1}\) es una solución al problema de valor inicial \(\dfrac{dy}{dx} =xy^2\), con \(y(0) =1/c^2\), en el intervalo \(-c <x < c\).
\end{enumerate}
% )))

\end{document}
