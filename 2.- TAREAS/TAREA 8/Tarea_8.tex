\documentclass[letterpaper,10pt]{memoir}

% === PAQUETES === (((
\usepackage[utf8]{inputenc}
\usepackage{amsfonts}
\usepackage{amsmath}
\usepackage{graphicx}
\usepackage{anysize} 
% )))

% === DATOS === (((
% \pdfinfo{%
	% /Title    (Tarea <++> de Ecuaciones Diferenciales)
	% /Author   (Sandra Elizabeth Delgadillo Alemán)
% }
\marginsize{1cm}{1cm}{1cm}{1cm} 
\pagestyle{empty}
% )))

% === TITULO === (((
\newcommand{\titulo}{
	\begin{minipage}{0.25\linewidth}
		\includegraphics[width= 0.9 \linewidth]{IMAGENES/log23.png}
	\end{minipage}
	\begin{minipage}{0.75\linewidth}
		\begin{center}
			\bfseries
			CENTRO DE CIENCIAS BÁSICAS \\
			DEPARTAMENTO DE MATEMÁTICAS Y FÍSICA \\
			ACADEMIA DE MATEMÁTICA AVANZADAS
		\end{center}
	\end{minipage}\\
	\begin{table}[ht]
		\centering
		\begin{tabular}{|*{3}{l|}p{3cm}|}
			\hline
			\textbf{Nombre del Estudiante:} & & \textbf{Fecha:} & \\ \hline
			\textbf{Materia:} & Ecuaciones Diferenciales & \textbf{Carrera:} &  \\ \hline
			\textbf{Profesor:} & Sandra Elizabeth Delgadillo Alemás & \textbf{Semestre:} & \\ \hline
			\textbf{Periodo:} & () Enero--Junio () Agosto--Diciembre & & \\ \hline
			\textbf{Tipo de Examen:} & Parcial: 1() \hspace{2mm} 2() \hspace{2mm} 3() & \textbf{Calificación:} & \\ \hline
		\end{tabular}
	\end{table}
} 
% )))

\begin{document}

\titulo

\section*{Ecuaciones Diferenciales Homogéneas (12).} % (((
\begin{enumerate}
	\item \textbf{(Hacer 4 ejercicios)} Determine la solución general de las siguientes ecuaciones diferenciales de segundo orden.
		\begin{enumerate}
			\item \(4 y'' +y \,' =0\). \textbf{Solución.} \(y=c_1+c_2e^{-x/4}\).
			\item \(y'' -y \,' -6y=0\). \textbf{Solución.} \(y=c_1e^{3x} +c_2e^{-2x}\).
			\item \(y'' +8y \,' +16y=0\). \textbf{Solución.} \(y=c_1e^{-4x} +c_2xe^{-4x}\).
			\item \(12y '' -5y \,' -2y=0\). \textbf{Solución.} \(y=c_1e^{2x/3} +c_2e^{-x/4}\).
			\item \(y'' +9y=0\). \textbf{Solución.} \(y=c_1 \cos 3x+c_2 \sin 3x\).
			\item \(y'' -4y \,' +5y=0\). \textbf{Solución.} \(y=e^{2x} (c_1 \cos x+c_2 \sin x)\).
			\item \(3y'' +2y \,' +y=0\). \textbf{Solución.} \(y=e^{-x/3} \Bigg(c_1 \cos \dfrac{\sqrt{2}}{3} x+c_2 \sin \dfrac{\sqrt{2}}{3} x\Bigg)\).
		\end{enumerate}
	\item \textbf{(Hacer 4 ejercicios)} Determine la solución general de las siguientes ecuaciones diferenciales de orden superior
		\begin{enumerate}
			\item \(y''' -4y'' -5y \,' =0\). \textbf{Solución.} \(y=c_1+c_2e^{-x} +c_3e^{5x}\).
			\item \(y''' -5y'' +3y \,' +9y=0\). \textbf{Solución.} \(y=c_1e^{-x} +c_2e^{3x} +c_3xe^{3x}\).
			\item \(\dfrac{d^3u}{dt^3} + \dfrac{d^2u}{dt^2} 2u=0\). \textbf{Solución.} \(u=c_1e^t+e^{-t} (c_2 \cos t+c_3 \sin t)\).
			\item \(y''' +3y'' +3y \,' +y=0\). \textbf{Solución.} \(y=c_1e^{-x} +c_2xe^{-x} +c_3x^2e^{-x}\).
			\item \(y^{(4)} +y''' +y'' =0\). \textbf{Solución.} \(y=c_1+c_2x+e^{-x/2} \Big(c_3 \cos \dfrac{\sqrt{3}}{2} x+c_4 \sin \dfrac{\sqrt{3}}{2} x\Big)\).
			\item \(16 \dfrac{d^4y}{dt^4} +24 \dfrac{d^2 y}{dt^2} +9y=0\). \textbf{Solución.} \(y=c_1 \cos \dfrac{\sqrt{3}}{2} t+c_2 \sin \dfrac{\sqrt{3}}{2} t+c_3t \cos \dfrac{\sqrt{3}}{2} t+c_4 t \sin \dfrac{\sqrt{3}}{2} t\).
			\item \(\dfrac{d^5u}{dr^5} +5 \dfrac{d^4 u}{dr^4} -2 \dfrac{d^3u}{dr^3} -10 \dfrac{d^2u}{dr^2} + \dfrac{du}{dr} +5u=0\). \textbf{Solución.} \(u=c_1e^r+c_2re^r+c_3e^{-r} +c_4re^{-r} +c_5e^{-5r}\).
		\end{enumerate}
	\item \textbf{(Hacer 2 ejercicios)} Resuelve los siguientes problemas de valor inicial y esboza su grafica, haciendo énfasis en la interpretación geométrica de las condiciones iniciales.
		\begin{enumerate}
			\item \(y'' +16y=0\), \(y(0) =2\), \(y \,' (0) =-2\). \textbf{Solución.} \(y=2 \cos 4x- \dfrac{1}{2} \sin 4x\).
			\item \(\dfrac{d^2y}{dt^2} -4 \dfrac{dy}{dt} -5y=0\), \(y(1) =0\), \(y \,' (1) =2\). \textbf{Solución.} \(y=- \dfrac{1}{3} e^{-(t-1)} + \dfrac{1}{3} e^{5(t-1)}\).
			\item \(y'' +y \,' +2y=0\), \(y(0) =0\), \(y \,' (0) =0\). \textbf{Solución.} \(y=0\).
			\item \(y''' +12y'' +36y' =0\), \(y(0) =0\), \(y \,' (0) =0\), \(y'' (0) =-7\). \textbf{Solución.} \(y=- \dfrac{7}{36} + \dfrac{7}{36} e^{-6x} + \dfrac{7}{6} xe^{-6x}\).
		\end{enumerate}
	\item \textbf{(Hacer 2 ejercicios)} Resuelve las siguientes ecuaciones diferenciales, sujeta a las condiciones de frontera indicadas y esboza su gráfica, haciendo énfasis en la interpretación geométrica de las condiciones de frontera.
		\begin{enumerate}
			\item \(y'' -10y' +25y=0\), \(y(0) =1\), \(y(1) =0\). \textbf{Solución.} \(y=e^{5x} -xe^{5x}\).
			\item \(y'' +y=0\), \(y' (0) =0\), \(y' \Bigg(\dfrac{\pi}{2}\Bigg) =2\). \textbf{Solución.} \(y=-2 \cos x\).
		\end{enumerate}
\end{enumerate}
*Puedes usar GeoGebra \url{https://www.geogebra.org/m/KGWhcAqc} o WolframAlpha \url{https://www.wolframalpha.com/} para esbozar o verificar las graficas de las soluciones de PVI o PVF.
% )))

\end{document}
