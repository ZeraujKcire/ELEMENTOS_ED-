\documentclass{beamer}

% === PAQUETES === (((
\usepackage[utf8]{inputenc}
\usepackage{amsfonts}
\usepackage{amsmath}
\usepackage{graphicx}
\usepackage{anysize} 
% )))

% === DATOS === (((
% \pdfinfo{%
	% /Title    (Tarea <++> de Ecuaciones Diferenciales)
	% /Author   (Sandra Elizabeth Delgadillo Alemán)
% }
\marginsize{1cm}{1cm}{1cm}{1cm} 
\pagestyle{empty}
% )))

% === TITULO === (((
\newcommand{\titulo}{
	\begin{minipage}{0.25\linewidth}
		\includegraphics[width= 0.9 \linewidth]{IMAGENES/log23.png}
	\end{minipage}
	\begin{minipage}{0.75\linewidth}
		\begin{center}
			\bfseries
			CENTRO DE CIENCIAS BÁSICAS \\
			DEPARTAMENTO DE MATEMÁTICAS Y FÍSICA \\
			ACADEMIA DE MATEMÁTICA AVANZADAS
		\end{center}
	\end{minipage}\\
	\begin{table}[ht]
		\centering
		\begin{tabular}{|*{3}{l|}p{3cm}|}
			\hline
			\textbf{Nombre del Estudiante:} & & \textbf{Fecha:} & \\ \hline
			\textbf{Materia:} & Ecuaciones Diferenciales & \textbf{Carrera:} &  \\ \hline
			\textbf{Profesor:} & Sandra Elizabeth Delgadillo Alemás & \textbf{Semestre:} & \\ \hline
			\textbf{Periodo:} & () Enero--Junio () Agosto--Diciembre & & \\ \hline
			\textbf{Tipo de Examen:} & Parcial: 1() \hspace{2mm} 2() \hspace{2mm} 3() & \textbf{Calificación:} & \\ \hline
		\end{tabular}
	\end{table}
} 
% )))

\begin{document}

\frame{\titlepage}

\begin{frame}[t]
	\frametitle{Teorema de Existencia y Unicidad.}
\begin{block}{Teorema de Existencia y Unicidad.}
	Considere el P.V.I. siguiente
	\[
		a_n(x) y^{(n)} + a_{n-1} (x) y^{(n-1)} + \;\cdots\; + a_1(x) y' +a_0(x) y = g(x).
	\]
	Sujeto a \(a_n(x) , a_{n-1} (x) , \;\cdots\; a_1(x) ,a_0(x)\) y \(g(x)\) continuas en el intervalo \(I\) y además, \(a_n(x) \ne 0\), \(\forall x \in I\). Si \(x=x_0\) en cualquier punto en el intervalo \(I\), entonces existe una única solución para el P.V.I. en \(I\).
\end{block}
	\begin{example}
		Determina si el P.V.I. \(x^2y'' -2xy' +2y=6\), \(y(0) =3\), \(y' (0) =1\), para \(x \in (- \infty , \infty)\) tiene solución única usando e T.E. y U.
	\end{example}
\end{frame}
\begin{frame}[t]
	\begin{exampleblock}{}
		\textbf{Solución.} Sean \(a_2(x) = x^2\), \(a_1(x) = -2x\), \(a_0(x) = 2\), \(g(x) = 6\) funciones continuas en el intervalo \((- \infty , \infty)\). \\[2mm]
		Pero
		\[
			\begin{array}{rcl}
				a_2(x) & = & x^2 = 0 \iff x = 0 \\[2mm]
				a_2(x) & = & x^2 \ne 0 \mbox{ en } (- \infty ,0) \mbox{ o } (0, \infty) \mbox{ pero éstos intervalos} \\[2mm]
				&& \mbox{no contienen a } x=0. \\[2mm]
				\therefore && \mbox{no se garantiza la existencia y unicidad de la} \\[2mm]
				&& \mbox{solución del P.V.I.}
			\end{array}
		\]
		Verifiquemos que \(y(x) = cx^2+x+3\) es solución del P.V.I. \(\forall c\). \\[2mm]
		Derivando:
		\[
			\begin{array}{rcl}
				y\,' & = & 2cx+1 \\[2mm]
				y\,'' & = & 2c.
			\end{array}
		\]
	\end{exampleblock}
\end{frame}

\begin{frame}[t]
	\begin{exampleblock}{}
		Sustiyuendo \(y\), \(y\,'\), \(y\,''\) en la E.D
		\[
			\begin{array}{rcl}
				x^2(2c) -2x(2cx+1) +2(c^2+x+3) & = & 6 \\[2mm]
				\iff \cancel{2cx^2} - \cancel{4cx^2} - \cancel{2x} + \cancel{2cx^2} + \cancel{2x} + 6 & = & 6 \\[2mm]
				6 & = & 6. \hspace{5mm}\forall c.
			\end{array}
		\]
		Veamos si \(y(x)\) satisface las C.I. \(y(0) =3\) y \(y\,' (0) =1\),
		\[
			\begin{array}{rcl}
				y(0) = c(0)^2+0+3 & = & 3 \\[2mm]
				3 & = & 3. \\[2mm]
				y\,' (0) = 2c(0) +1 & = & 1 \\[2mm]
				1 & = & 1.
			\end{array} \hspace{1cm} \forall c.
		\]
	\end{exampleblock}
\end{frame}

\begin{frame}[t]
	\begin{exampleblock}{}
		\begin{figure}[hbt!]
			\centering
			\includegraphics[width= 0.7 \linewidth]{IMAGENES/6/tikz.pdf}
		\end{figure}
		Al igual que los P.V.I., los P.V.F. pueden tener muchas soluciones, sólo una, o ninguna.
	\end{exampleblock}
\end{frame}

\begin{frame}[t]
	\begin{example}
		Considere la E.D. \(x'' +16x=0\), y \(x(t) = c_1 \cos 4t+c_2 \sin 4t\), la solución general explícita de dicha E.D.
		Encuentra la solución del problema de valor en la frontera cuyas condiciones son:
		\begin{enumerate}
			\item \(x(0) =0\), \(x(\pi /2) =0\).
			\item \(x(0) =0\), \(x(\pi /2) =1\).
			\item \(x(0) =0\), \(x(\pi /8) =0\).
		\end{enumerate}
		\textbf{Solución.} 
		\begin{enumerate}
			\item Apliquemos la C.F. en la solución general
		\[
			\begin{array}{rcl}
				x(0) = c_1 \cancelto{1}{\cos 4(0)} +c_2 \cancelto{0}{\sin 4(0)} & = & 0 \\[2mm]
				\iff c_1 & = & 0.\\[2mm]
				x(\pi /2) = c_1 \cos (4 \pi /2) + c_2 \sin (4 \pi /2) & = & 0 \\[2mm]
				\iff c_1 \cancelto{1}{\cos (2 \pi)} + c_2 \cancelto{0}{\sin (2 \pi)} & = & 0 \\[2mm]
				\iff c_1 & = & 0. \hspace{1cm} c_2 \in \mathbb{R}.
			\end{array}
		\]
		\end{enumerate}
	\end{example}
\end{frame}

\begin{frame}[t]
	\begin{exampleblock}{}
		\(\therefore \hspace{5mm} x(t) = c_2 \sin 4t\) es una familia de soluciones que satisfacen el P.V.F. Es decir, hay un número infinito de soluciones.
		\begin{figure}[hbtp!]
			\centering
			\includegraphics[width= 0.5 \linewidth]{IMAGENES/7/tikz.pdf}
		\end{figure}
		\begin{enumerate}
			\setcounter{enumi}{1}
		\item \vspace{-5mm}
			\[
				\begin{array}{rcl}
					x(0) = c_1 \cancelto{1}{\cos 4(0)} + c_2 \cancelto{0}{\sin 4(0)} & = & 0 \\[2mm]
					\iff c_1 & = & 0.\\[2mm]
					x(\pi /2) = c_1 \cos (4 \pi /2) + c_2 \sin (4 \pi /2) & = & 1 \\[2mm]
					\iff c_1 \cancelto{1}{\cos (2 \pi)} + c_2 \cancelto{0}{\sin (2 \pi)} & = & 1 \\[2mm]
					c_1 & = & 1.
				\end{array}
			\]
		\end{enumerate}
	\end{exampleblock}
\end{frame}

\begin{frame}[t]
	\begin{exampleblock}{}
		\begin{enumerate}
			\setcounter{enumi}{2}
		\item Aplicando las condiciones \(x(0) = 0\), \(x(\pi /8) =0\),
			\[
				\begin{array}{rcl}
					x(0) = c_1 \cancelto{1}{\cos 4(0)} + c_2 \cancelto{0}{\sin 4(0)} & = & 0 \\[2mm]
					\iff c_1 & = & 0. \\[2mm]
					x(\pi /8) = c_1 \cancelto{0}{\cos (4 \pi /8)} + c_2 \cancelto{1}{\sin (4 \pi /8)} & = & 0 \\[2mm]
					\iff c_2 & = & 0. \\[2mm]
					\therefore \hspace{5mm} x(t) & = & 0.
				\end{array}
			\]
				\begin{figure}[hbtp!]
					\centering
					\includegraphics[width= 0.6 \linewidth]{IMAGENES/8/tikz.pdf}
				\end{figure}
		\end{enumerate}
	\end{exampleblock}
\end{frame}

\end{document}
