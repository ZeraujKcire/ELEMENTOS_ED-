\documentclass{beamer}

% === PAQUETES === (((
\usepackage[utf8]{inputenc}
\usepackage{amsfonts}
\usepackage{amsmath}
\usepackage{graphicx}
\usepackage{anysize} 
% )))

% === DATOS === (((
% \pdfinfo{%
	% /Title    (Tarea <++> de Ecuaciones Diferenciales)
	% /Author   (Sandra Elizabeth Delgadillo Alemán)
% }
\marginsize{1cm}{1cm}{1cm}{1cm} 
\pagestyle{empty}
% )))

% === TITULO === (((
\newcommand{\titulo}{
	\begin{minipage}{0.25\linewidth}
		\includegraphics[width= 0.9 \linewidth]{IMAGENES/log23.png}
	\end{minipage}
	\begin{minipage}{0.75\linewidth}
		\begin{center}
			\bfseries
			CENTRO DE CIENCIAS BÁSICAS \\
			DEPARTAMENTO DE MATEMÁTICAS Y FÍSICA \\
			ACADEMIA DE MATEMÁTICA AVANZADAS
		\end{center}
	\end{minipage}\\
	\begin{table}[ht]
		\centering
		\begin{tabular}{|*{3}{l|}p{3cm}|}
			\hline
			\textbf{Nombre del Estudiante:} & & \textbf{Fecha:} & \\ \hline
			\textbf{Materia:} & Ecuaciones Diferenciales & \textbf{Carrera:} &  \\ \hline
			\textbf{Profesor:} & Sandra Elizabeth Delgadillo Alemás & \textbf{Semestre:} & \\ \hline
			\textbf{Periodo:} & () Enero--Junio () Agosto--Diciembre & & \\ \hline
			\textbf{Tipo de Examen:} & Parcial: 1() \hspace{2mm} 2() \hspace{2mm} 3() & \textbf{Calificación:} & \\ \hline
		\end{tabular}
	\end{table}
} 
% )))

\begin{document}

\frame{\titlepage}

\begin{frame}[t]
	\frametitle{Problemas de Valor Inicial.}
	\begin{block}{}
		En esta unidad estudiaremos las ecuaciones diferenciales lineales de orden \(n\) (\(n>1\)) las cuales tienen la siguiente forma:
		\begin{equation}
			a_n(x) y^{(n)} + \;\cdots\; + a_1(x) y' +a_0(x) y=g(x).
			\label{eq:definicion_lineal}
		\end{equation}
		donde \(a_n, \;\ldots,\; a_1,a_0,g\), son funciones de valores real en un intervalo \(I\).
	\end{block} \vspace{5mm}
	\begin{block}{Problema de Valor Inicial}
		\textbf{Definición.} Un \textbf{P.V.I. de orden \(n\)} consiste en resolve la ecuación diferncial (\ref{eq:definicion_lineal}) sujeta a las condiciones
		\[
			y(x_0) = y_0, \;\cdots\; y^{(n-1)} (x_0) = y_{n-1}.
		\]
		donde: \(y_0,y_1, \;\cdots\; y_{n-1}\) son constante reales dadas.
	\end{block}
\end{frame}

\begin{frame}[t]
	\begin{block}{}
		\textbf{Interpretación geométrica:} P.V.I. de \(2^{do}\) Orden.
		\[
			\begin{array}{rcl}
				a_2(x) y'' +a_1(x) y' +a_0(x) y & = & g(x) \\[2mm]
				y(x_0) = y_0 , \hspace{5mm} y' (x_0) & = & y_1 >0.
			\end{array}
		\]
		\begin{figure}[ht]
			\centering
			\includegraphics[width= 0.6 \linewidth]{IMAGENES/1/tikz.pdf}
		\end{figure}
	\end{block}
\end{frame}

\begin{frame}[t]
	\begin{block}{}
		\textbf{Interprteación geométrica:} P.V.I. de \(3^{er}\) Orden.
		\[
			\begin{array}{rcl}
				3y''' +5y'' -y' +7y & = & 0 \\[2mm]
				y(1) =0 \hspace{5mm} y'(1) =0 m y'' (1) & = & 0 \\[2mm]
			\end{array}
		\]
		\begin{figure}[ht]
			\centering
			\includegraphics[width= 0.7 \linewidth]{IMAGENES/2/tikz.pdf}
		\end{figure}
	\end{block}
\end{frame}

\begin{frame}[t]
	\frametitle{Problemas de Valor en la Frontera.}
	\begin{block}{Problema de Valor en la Frontera.}
		Un \textbf{P.V.F. de orden \(n\)} consiste en resolver la ecuación diferencial (\ref{eq:definicion_lineal}) sujeto a condiciones sobre la variable dependiente \(y\) (o sus derivadas) en puntos distintos de la variable independiente, esto es
		\[
			y(x_0) =y_0 \;,\;  y(x_1) =y_1 \;, \;\ldots,\; y(x_{n-1}) =y_{n-1}. \vspace{-2.5mm}
		\]
		donde \(x_0,x_1, \;\ldots,\; x_{n-1} \in I\), y \(y_0,y_1, \;\ldots,\; y_{n-1}\) son constantes dadas.\\[-5.5mm]
		\begin{minipage}{0.3\linewidth}
			En este caso, se busca una solución de la E.D. que pase por los puntos \((x_0,y_0) , (x_1,y_1) , \;\ldots,\)\\\( (x_n,y_n)\).
		\end{minipage} \hspace{5mm}
		\begin{minipage}{0.6\linewidth}
			\includegraphics[width= 0.9\linewidth]{IMAGENES/3/tikz.pdf}
		\end{minipage}
	\end{block}
\end{frame}

\begin{frame}[t]
	\begin{block}{}
	\textbf{Interpretación geométrica:} P.V.F. de \(2^{do}\) Orden.
	\[
		\begin{array}{rcl}
			a_2(x) y'' +a_1(x) y' + a_0(x) y & = & g(x) \\[2mm]
			y(x_0) = y_0 , \hspace{4mm} y(x_1) & = & y_1. \vspace{-1cm}
		\end{array}
	\]
		\begin{figure}[ht]
			\centering
			\includegraphics[width= 0.55 \linewidth]{IMAGENES/4/tikz.pdf}
		\end{figure}
	\end{block}
\end{frame}

\begin{frame}[t]
	\begin{block}{}
		Otro tipo de condiciones de frontera son:\\
		\begin{minipage}{0.3\linewidth}
			\begin{enumerate}
					\footnotesize 
				\item \(y(x_0) =y_0\), \(y' (x_1) =y_1 < 0.\)
				\item \(y'(x_0)=y_0\), \(y(x_1) =y_1\).
				\item \(y' (x_0) =y_0\), \(y'(x_0) =y_1 >0\).
			\end{enumerate}
		\end{minipage}\hspace{5mm}
		\begin{minipage}{0.6\linewidth}
			\includegraphics[width= \linewidth]{IMAGENES/5/tikz.pdf}
		\end{minipage}
	\end{block}
\end{frame}

\end{document}
