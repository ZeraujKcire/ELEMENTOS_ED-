\documentclass{beamer}

% === PAQUETES === (((
\usepackage[utf8]{inputenc}
\usepackage{amsfonts}
\usepackage{amsmath}
\usepackage{graphicx}
\usepackage{anysize} 
% )))

% === DATOS === (((
% \pdfinfo{%
	% /Title    (Tarea <++> de Ecuaciones Diferenciales)
	% /Author   (Sandra Elizabeth Delgadillo Alemán)
% }
\marginsize{1cm}{1cm}{1cm}{1cm} 
\pagestyle{empty}
% )))

% === TITULO === (((
\newcommand{\titulo}{
	\begin{minipage}{0.25\linewidth}
		\includegraphics[width= 0.9 \linewidth]{IMAGENES/log23.png}
	\end{minipage}
	\begin{minipage}{0.75\linewidth}
		\begin{center}
			\bfseries
			CENTRO DE CIENCIAS BÁSICAS \\
			DEPARTAMENTO DE MATEMÁTICAS Y FÍSICA \\
			ACADEMIA DE MATEMÁTICA AVANZADAS
		\end{center}
	\end{minipage}\\
	\begin{table}[ht]
		\centering
		\begin{tabular}{|*{3}{l|}p{3cm}|}
			\hline
			\textbf{Nombre del Estudiante:} & & \textbf{Fecha:} & \\ \hline
			\textbf{Materia:} & Ecuaciones Diferenciales & \textbf{Carrera:} &  \\ \hline
			\textbf{Profesor:} & Sandra Elizabeth Delgadillo Alemás & \textbf{Semestre:} & \\ \hline
			\textbf{Periodo:} & () Enero--Junio () Agosto--Diciembre & & \\ \hline
			\textbf{Tipo de Examen:} & Parcial: 1() \hspace{2mm} 2() \hspace{2mm} 3() & \textbf{Calificación:} & \\ \hline
		\end{tabular}
	\end{table}
} 
% )))

\begin{document}

\frame{\titlepage}

\begin{frame}[t]
	\frametitle{Teorema de Existencia y Unicidad.}
\begin{block}{Teorema de Existencia y Unicidad.}
	Considere el P.V.I. siguiente
	\[
		a_n(x) y^{(n)} + a_{n-1} (x) y^{(n-1)} + \;\cdots\; + a_1(x) y' +a_0(x) y = g(x).
	\]
	Sujeto a \(a_n(x) , a_{n-1} (x) , \;\cdots\; a_1(x) ,a_0(x)\) y \(g(x)\) continuas en el intervalo \(I\) y además, \(a_n(x) \ne 0\), \(\forall x \in I\). Si \(x=x_0\) en cualquier punto en el intervalo \(I\), entonces existe una única solución para el P.V.I. en \(I\).
\end{block}
	\begin{example}
		Determina si el P.V.I. \(x^2y'' -2xy' +2y=6\), \(y(0) =3\), \(y' (0) =1\), para \(x \in (- \infty , \infty)\) tiene solución única usando e T.E. y U.
	\end{example}
\end{frame}
\begin{frame}[t]
\end{frame}

\begin{frame}[t]
	\begin{minipage}{0.6\linewidth}
		\includegraphics[width= \linewidth]{IMAGENES/6/tikz.pdf}
	\end{minipage}
	\begin{minipage}{0.3\linewidth}
	\end{minipage}
\end{frame}

\begin{frame}[t]
	\begin{example}
		Considere la siguiente E.D. \(x'' +16x=0\), y \(x(t) = c_1 \cos 4t+c_2 \sin 4t\), la solución general explícita de dicha E.D. Entonces la solución del problema en la frontera cuyas condiciones son:
		\begin{enumerate}
			\item \(x(0) =0\), \(x(\pi /2) =0\).
			\item \(x(0) =0\), \(x(\pi /2) =1\).
			\item \(x(0) =0\), \(x(\pi /8) =0\).
		\end{enumerate}
	\end{example}
\end{frame}

\end{document}
