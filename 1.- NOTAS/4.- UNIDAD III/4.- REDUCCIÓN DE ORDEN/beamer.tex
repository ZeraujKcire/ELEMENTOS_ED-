\documentclass{beamer}

% === PAQUETES === (((
\usepackage[utf8]{inputenc}
\usepackage{amsfonts}
\usepackage{amsmath}
\usepackage{graphicx}
\usepackage{anysize} 
% )))

% === DATOS === (((
% \pdfinfo{%
	% /Title    (Tarea <++> de Ecuaciones Diferenciales)
	% /Author   (Sandra Elizabeth Delgadillo Alemán)
% }
\marginsize{1cm}{1cm}{1cm}{1cm} 
\pagestyle{empty}
% )))

% === TITULO === (((
\newcommand{\titulo}{
	\begin{minipage}{0.25\linewidth}
		\includegraphics[width= 0.9 \linewidth]{IMAGENES/log23.png}
	\end{minipage}
	\begin{minipage}{0.75\linewidth}
		\begin{center}
			\bfseries
			CENTRO DE CIENCIAS BÁSICAS \\
			DEPARTAMENTO DE MATEMÁTICAS Y FÍSICA \\
			ACADEMIA DE MATEMÁTICA AVANZADAS
		\end{center}
	\end{minipage}\\
	\begin{table}[ht]
		\centering
		\begin{tabular}{|*{3}{l|}p{3cm}|}
			\hline
			\textbf{Nombre del Estudiante:} & & \textbf{Fecha:} & \\ \hline
			\textbf{Materia:} & Ecuaciones Diferenciales & \textbf{Carrera:} &  \\ \hline
			\textbf{Profesor:} & Sandra Elizabeth Delgadillo Alemás & \textbf{Semestre:} & \\ \hline
			\textbf{Periodo:} & () Enero--Junio () Agosto--Diciembre & & \\ \hline
			\textbf{Tipo de Examen:} & Parcial: 1() \hspace{2mm} 2() \hspace{2mm} 3() & \textbf{Calificación:} & \\ \hline
		\end{tabular}
	\end{table}
} 
% )))

\begin{document}

\frame{\titlepage}

\begin{frame}[t]
	\frametitle{Reducción de Orden.}
	\begin{block}{}
		Uno de los hechos más interesantes al estudiar a ecuaciones diferenciales de \(2^{do}\) orden, es que podemos formar una \(2^{da}\) solución \(y_2(x)\) de la E.D.H
		\[
			a_2(x) y'' +a_1(x) +a_0(x) y =0 \hspace{5mm} a_2(x) \ne 0 \mbox{ en } I.
		\]
		Siempre y cuando se conozca una solución \(y_1(x)\) no trivial (es decir, \(y_1(x) \ne 0\)) en \(I\), de tal forma que \(\{y_1,y_2\}\) sea linealmente independiente. \\[2mm]
		Observemos que dos soluciones son linealmente dependientes si una es un múltiplo escalar de la otra.
		Dados \(y_1,y_2\) son L.D. si y sólo si \(y_2(x) = cy_1(x)\).
		\[
			\iff \dfrac{y_2(x)}{y_1(x)} = c \hspace{5mm} \forall x \in I.
		\]
	\end{block}
\end{frame}

\begin{frame}[t]
	\begin{block}{}
		Ahora, estamos interesados en determinar \(y_2(x)\) tal que sea L.I: a \(y_1(x)\), por lo cual, proponemos \(\dfrac{y_2(x)}{y_1(x)} = \mu (x)\)
		\[
			\iff y_2(x) = \mu (x) y_1(x).
		\]
		Derivemos \(y_2(x) = \mu (x) y_1(x)\)
		\[
			y'_2 (x) = \mu ' y_1+ \mu y_1' \hspace{7mm} 
			\begin{array}{rcl}
				y'' & = & \mu '' y_1+ \mu 'y_1' + \mu 'y_1' + \mu y'' \\[2mm]
				& = & \mu '' y_1+2 \mu 'y_1' + \mu y'' .
			\end{array}
		\]
		ahora sustituyendo \(y_2\), \(y_2'\), \(y_2''\) en la E.D. en su forma estándar \(y'' + p(x) y' + q(x) y =0\), donde
		\[
			p(x) = \dfrac{a_1(x)}{a_2(x)} \;,\; q(x) = \dfrac{a_0(x)}{a_2(x)}.
		\]
	\end{block}
\end{frame}

\begin{frame}[t]
	\begin{block}{}
		esto es
		\footnotesize 
		\[
			\begin{array}{rcl}
				(\mu '' y_1+2 \mu 'y_1' + \mu y'') +p(x) (y'_2 (x) + \mu ' y_1+ \mu y_1') +q(x) (\mu y_1) & = & 0 \\[2mm]
				(\mu y_1'' +p(x) \mu y_1' +q(x) \mu y_1) + (\mu '' y_1+2 \mu 'y_1' +p(x) \mu ' y_1) & = & 0 \\[2mm]
				\mu \cancel{(y_1'' +p(x) y_1' +q(x) y_1)} + (\mu '' +2 \mu 'y_1' +p(x) \mu ' y_1) & = & 0 \\[2mm]
				\textcolor{red}{ \underline{\textcolor{black}{\mu '' y_1+ \mu ' (2y_1' +p(x) y_1) =0}} \hspace{5mm} E.D. \;\; \mu = \mu (x)} &&
			\end{array}
		\]
		Hagamos un cambio de variable
		\[
			\tilde{\mu} = \mu ' \;\implies\; \tilde{\mu} ' = \mu ''.
		\]
		Luego:
		\[
			\begin{array}{rcl}
				\tilde{\mu} ' y_1 + \tilde{\mu} (2y_1' +p(x) y_1) & = & 0 \\[2mm]
				\tilde{\mu} ' y_1+ \tilde{\mu} (2y_1' +p(x) y_1) & = & 0 \\[2mm]
				\iff y_1 \tilde{\mu} ' & = & -(2y_1' +p(x) y_1) \tilde{\mu} \\[2mm]
				\iff \dfrac{\tilde{\mu} '}{\tilde{\mu}} & = & - \dfrac{2y_1' +p(x) y_1}{y_1}.
			\end{array}
		\]
	\end{block}
\end{frame}

\begin{frame}[t]
	\begin{block}{}
		Integramos con respecto a \(x\),
		\small
		\[
			\begin{array}{rcl}
				\dis\int \dfrac{\tilde{\mu} '}{\mu} dx & = & - \dis\int \dfrac{2y_1' +p(x) y_1}{y_1} dx = - \Bigg[2 \dis\int \dfrac{y_1'}{y_1} dx + \dis\int p(x) dx\Bigg] \\[5mm]
				\iff \ln | \tilde{\mu} | & = & -2 \ln |y_1| - \dis\int p(x) dx = \ln |y_1|^{-2} - \dis\int p(x) dx+C \\[5mm]
				e^{\ln | \tilde{\mu} |} & = & e^{\ln |y_1|^{-2} - \int p(x) dx +C} = e^{\ln |y_1|^{-2}} \cdot e^{-\int p(x) dx} \cdot e^C \\[5mm]
				\tilde{\mu} & = & C \big| y_1 \big| ^{-2} e^{-\int p(x) dx} . \vspace{-2mm}
			\end{array}
		\]
		Como \(\tilde{\mu} = \mu '\), entonces obtenemos a \(\mu\) integrando. \vspace{-2mm} 
		\[
			\mu (x) = \dis\int \mu ' dx = \dis\int c \big| y_1 \big| ^{-2} e^{-\int p(x) dx} dx.
		\]
		\begin{center}
			\color{red} \fbox{\color{black} \(\tilde{\mu} = \dis\int \dfrac{e^{-\int p(x) dx}}{y_1^2} dx\)} \hspace{2mm} con \(c=1\).
		\end{center} 
	\end{block}
\end{frame}

\begin{frame}[t]
	\begin{block}{}
		Luego, dada \(y_1 \ne 0\) solución de la E.D., una \(2^{da}\) solción linealmente independiente a \(y_1\) está dada por
		\[
			y_2(x) = \mu (x) y_1(x) \hspace{3mm} \mbox{donde } \mu (x) = \dis\int \dfrac{e^{-\int p(x) dx}}{y_1^2}dx.
		\]
		Así pues, \(\{y_1,y_2\}\) conforman una c.f.s y la solución general de la E.D. está dado por la combinación lineal de éstas.
	\end{block} \vspace{2mm}
	\begin{example}
		Sea \(y_1(x) = x^2\) una solución de la E.D. \(x^2y'' -2xy' +4y=0\). Determine la solución general de la E.D. en el intervalo \(I=(0, \infty)\).
	\end{example}
\end{frame}
\begin{frame}[t]
\end{frame}

\begin{frame}[t]
	\begin{alertblock}{Ejercicio.}
		Si \(y_1(x) = e^x\) es solución de \(y'' -y=0\), en el intervalo \((- \infty , \infty)\). Aplique reducción de orden para determinar \(y_2(x)\), L.I. a \(y_1(x)\) y determine explícitamente la solución general de la E.D. dadas.
	\end{alertblock}
\end{frame}

\end{document}
