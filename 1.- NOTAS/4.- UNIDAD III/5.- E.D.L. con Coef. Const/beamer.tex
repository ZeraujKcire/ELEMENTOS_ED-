\documentclass{beamer}

% === PAQUETES === (((
\usepackage[utf8]{inputenc}
\usepackage{amsfonts}
\usepackage{amsmath}
\usepackage{graphicx}
\usepackage{anysize} 
% )))

% === DATOS === (((
% \pdfinfo{%
	% /Title    (Tarea <++> de Ecuaciones Diferenciales)
	% /Author   (Sandra Elizabeth Delgadillo Alemán)
% }
\marginsize{1cm}{1cm}{1cm}{1cm} 
\pagestyle{empty}
% )))

% === TITULO === (((
\newcommand{\titulo}{
	\begin{minipage}{0.25\linewidth}
		\includegraphics[width= 0.9 \linewidth]{IMAGENES/log23.png}
	\end{minipage}
	\begin{minipage}{0.75\linewidth}
		\begin{center}
			\bfseries
			CENTRO DE CIENCIAS BÁSICAS \\
			DEPARTAMENTO DE MATEMÁTICAS Y FÍSICA \\
			ACADEMIA DE MATEMÁTICA AVANZADAS
		\end{center}
	\end{minipage}\\
	\begin{table}[ht]
		\centering
		\begin{tabular}{|*{3}{l|}p{3cm}|}
			\hline
			\textbf{Nombre del Estudiante:} & & \textbf{Fecha:} & \\ \hline
			\textbf{Materia:} & Ecuaciones Diferenciales & \textbf{Carrera:} &  \\ \hline
			\textbf{Profesor:} & Sandra Elizabeth Delgadillo Alemás & \textbf{Semestre:} & \\ \hline
			\textbf{Periodo:} & () Enero--Junio () Agosto--Diciembre & & \\ \hline
			\textbf{Tipo de Examen:} & Parcial: 1() \hspace{2mm} 2() \hspace{2mm} 3() & \textbf{Calificación:} & \\ \hline
		\end{tabular}
	\end{table}
} 
% )))

\begin{document}

\frame{\titlepage}

\section{E.D. Homogéneas con Coeficientes Constantes.} % (((
\begin{frame}[t]
	\frametitle{E.D. Lineales Homgéneas con Coeficientes Constantes.}
	\begin{block}{}
		Recordemos que una E.D. lineal homogénea de \(1{er}\) orden con coeficientes constantes está dada por
		\[
			\begin{array}{rcl}
				a_1y' +a_0y & = & 0 \;,\; a_1 \ne 0 \\[2mm]
				\iff y' + \dfrac{a_0}{a_1} y & = & \dfrac{0}{a_1} =0 \\[2mm]
			\end{array}
		\]
		\begin{center}
			\(\iff\) \color{red} \(\underbrace{\fbox{\color{black} \(y' + \alpha y = 0\)}} _\text{Forma Estándar}\)  \color{black} . \hspace{2mm} \(\alpha = a_0/a_1\)
		\end{center} 
		Resolvemos la E.D. usando un factor integrante
		\[
			\mu (x) = e^{\int p(x) dx} = e^{\int \alpha dx} = e^{\alpha x}.
		\]
		Multipliquemos la E.D. por \(\mu (x)\).
	\end{block}
\end{frame}

\begin{frame}[t]
	\begin{block}{}
		\[
			\begin{array}{rcl}
				e^{\alpha x} y' + e^{\alpha x} y & = & e^{\alpha x} \cdot 0 =0 \\[2mm]
				\iff \dfrac{d}{dx} \big(e^{\alpha x} y\big) & = & 0
			\end{array}
		\]
		Integramos \vspace{-5mm}
		\[
			\begin{array}{c}
				\dis\int \dfrac{d}{dx} \big(e^{\alpha x} y\big) = \dis\int 0dx \\[5mm]
				\iff e^{\alpha x} y=c \iff y(x) = ce^{- \alpha x}.
			\end{array}
		\]
		Por lo anterior, es natural tratar de determinar si existen soluciones exponenciales para las E.D.L.H. de \(2^{do}\) orden con coeficientes constates de la forma
		\begin{center}
			\color{red} \fbox{\color{black} \(ay'' +by' +cy=0\)} \color{black} \hspace{5mm} \(a,b,c \in \mathbb{R}\).
		\end{center} 
		Veamos si ésta E.D. tiene soluciones exponenciales.
	\end{block}
\end{frame}

\begin{frame}[t]
	\begin{block}{}
		\[
			\begin{array}{rcl}
				y(x) & = & e^{mx} \\[2mm]
				y' (x) & = & me^{mx} \\[2mm]
				y'' (x) & = & m^2 e^{mx}.
			\end{array}
		\]
		Sustituyendo en la E.D.
		\[
			\begin{array}{rcl}
				a(m^2e^{mx}) +b(me^{mx}) +c(e^{mx}) & = & 0 \\[2mm]
				am^2e^{mx} +bme^{mx} +ce^{mx} & = & 0 \\[2mm]
				e^{mx} (am^2+bm+c) & = & 0.
			\end{array}
		\]
		Como \(e^{mx} \ne 0\), \(\forall x \in \mathbb{R} \;\implies\;\) \color{red} \(\underbrace{\fbox{\color{black} \(am^2+bm+c=0\)}} _\text{Ecuación Característica}\) \\[2mm]
		\color{black} Luego, \(y(x) = e^{mx}\) es solución de la E.D. siempre y cuando \(m\) sea una raíz de la ecuación característica.
	\end{block}
\end{frame}

\begin{frame}[t]
	\begin{block}{}
		La ecuación característica se puede resolver factorizando, o por fórmula general. En este caso las raíces están dadas por
		\[
			\begin{array}{c}
				m_{1,2} = \dfrac{-b \pm  \sqrt{b^2-4ac}}{2a} \\[2mm]
				m_1 = \dfrac{-b}{2a} + \dfrac{\sqrt{b^2-4ac}}{2a} \hspace{5mm} m_2 = \dfrac{-b}{2a} - \dfrac{\sqrt{b^2-4ac}}{2a}.
			\end{array}
		\]
		A \(b^2-4ac\) se le conoce como el \textbf{discriminante}. Se tienen \(3\) casos dependiendo de lo siguiente.
		\begin{enumerate}
			\item \(b^2-4ac >0\), dos raíces reales distintas \(m_1 \ne m_2\).
			\item \(b^2-4ac=0\), dos raíces reales e iguales, \(m_1=m_2 = -b/2a\).
			\item \(b^2-4ac<0\), raíces imaginarias conjugadas.
				\[
					\begin{array}{rcl}
						m_1 & = & d+ie \\
						m_2 & = & d-ie
					\end{array} \hspace{5mm} d,e \in \mathbb{R}.
				\]
		\end{enumerate}
	\end{block}
\end{frame}

\begin{frame}[t]
	\begin{block}{Caso 1. Raíces reales y distintas, \(m_1 \ne m_2\).}
		En este caso, se tienen dos soluciones de la E.D. \(y_1(x) = e^{m_1x}\), y \(y_2(x) =e^{m_2x}\). Veamos que \(y_1,y_2\) son L.I.
		\[
			\begin{array}{rcl}
				W[y_1,y_2] (x) & = & \begin{vmatrix}
					e^{m_2x} & e^{m_2x} \\
					m_1e^{m_1x} & m_2e^{m_2x}
				\end{vmatrix}\\[2mm]
				& = & e^{m_1x} m_2e^{m_2x} -m_1e^{m_1x} e^{m_2x} \\[2mm]
				& = & m_2e^{m_1x+m_2x} -m_1e^{m_1x+m_2x} \\[2mm]
				& = & e^{(m_1+m_2) x} (m_2-m_1) \ne 0.
			\end{array}
		\]
		ya que \(e^{(m_1+m_2) x} >0\), \(\forall x\), y \(m_1 \ne m_2\).
		\[
			\therefore \hspace{5mm} y_1 \mbox{ y } y_2 \mbox{ son L.I.}
		\]
		Luego, \(y_1\) y \(y_2\) conforman un conjunto fundamental de soluciones de la E.D. 
	\end{block}
\end{frame}

\begin{frame}[t]
	\begin{block}{}
		Por lo tanto, la solución general está dada por
		\[
			\color{red} \underbrace{\color{black} y(x) = c_1e^{m_1x} + c_2e^{m_2 x}} \color{black} \hspace{5mm} c_1,c_1 \mbox{ cts. arb.}
		\]
	\end{block} \vspace{7mm}
	\begin{block}{Caso 2. Raíces reales e iguales \(m_1=m_2 =- \frac{b}{2a}\).}
		En este caso, se tiene una solución de la E.D. \(y_1(x) = e^{- \frac{b}{2a}  x}\). Usamos reducción de orden para obtener una segunda solución L.I. a \(y_1(x)\).
		\[
			y_2(x) = \mu (x) y_1(x) \mbox{ donde } \mu (x) = \dis\int \dfrac{e^{-\int p(x) dx}}{y_1^2(x)} dx.
		\]
		\(ay'' +by' +cy=0 \;\implies\; y'' + \dfrac{b}{a} y' + \dfrac{c}{a} y=0\),
	\end{block}
\end{frame}

\begin{frame}[t]
	\begin{block}{}
		\[
			\begin{array}{rcl}
				\mu (x)  & = & \dis\int \dfrac{e^{-\int b/a\;dx}}{\big(e^{- \frac{b}{2a} x}\big) ^2} dx = \dis\int \dfrac{e^{-b/a \;x}}{e^{-b/a\;x}} dx = \dis\int dx \\[2mm]
				\therefore \hspace{5mm} \mu (x) & = & x.
			\end{array}
		\]
		Entonces:
		\[
			y_2(x) = xe^{- \frac{b}{2a} x}.
		\]
		Verifiquemos que \(y_2(x)\) también es solución:
		\[
			\begin{array}{rcl}
				y_2' (x) & = & - \dfrac{b}{2a} xe^{- \frac{b}{2a} x} + e^{- \frac{b}{2a} x}. \\[2mm]
				y_2''(x) & = & \bigg(- \dfrac{b}{2a}\bigg) ^2xe^{- \frac{b}{2a} x} - \dfrac{b}{2a} e^{- \frac{b}{2a} x} = \dfrac{b^2}{4a^2} xe^{- \frac{b}{2a} x} - \dfrac{b}{a} e^{- \frac{b}{2a} x}.
			\end{array}
		\]
		Sustituyendo en la E.D.
	\end{block}
\end{frame}

\begin{frame}[t]
	\begin{block}{}
		\footnotesize 
		\[
			\begin{array}{rcl}
				a \bigg(\dfrac{b^2}{4a^2} xe^{- \frac{b}{2a} x} - \dfrac{b}{a} e^{- \frac{b}{2a} x}\bigg) + b \bigg(- \dfrac{b}{2a} xe^{- \frac{b}{2a} x} + e^{- \frac{b}{2a} x}\bigg) +c \big(xe^{- \frac{b}{2a} x}\big) & = & 0 \\[2mm]
				\dfrac{b^2}{4a} xe^{- \frac{b}{2a} x} - b e^{- \frac{b}{2a} x} - \dfrac{b^2}{2a} xe^{- \frac{b}{2a} x} +be^{- \frac{b}{2a} x} +cx e^{- \frac{b}{2a} x} & = & 0 \\[2mm]
				\iff - \dfrac{b^2}{4a} xe^{- \frac{b}{2a} x} +cxe^{- \frac{b}{2a} x} & = & 0 \\[2mm]
				\iff \forall x, \hspace{5mm} xe^{- \frac{b}{2a} x} \bigg(- \dfrac{b^2}{4a} +c\bigg) & = & 0.\\
				- \dfrac{b^2}{4a} +c=0 \;\; \iff \;\; b^2-4ac=0. \hspace{2cm} &&\\[2mm]
				\therefore \hspace{5mm} y_2 \mbox{ es solución de la E.D.} \hspace{2cm} &&
			\end{array}
		\]
		Ahora, vamos que son L.I. \vspace{-5mm}
		\[
			\begin{array}{rcl}
				W[y_1,y_2] (x) & = & \begin{vmatrix}
					e^{- \frac{b}{2a} x} & xe^{- \frac{b}{2a} x} \\
					\frac{b}{2a} xe^{-b/a} & - \frac{b}{2a} xe^{- \frac{b}{2a} x} +e^{- \frac{b}{2a} x}
				\end{vmatrix}\\
				& = & - \dfrac{b}{2a} xe^{-b/a} +e^{-b/a\;x} + \dfrac{b}{2a} xe^{-b/a\;x} \\[3mm]
				& = & e^{-b/a\;x} \ne 0, \hspace{5mm} \forall x.
			\end{array}
		\]
	\end{block}
\end{frame}

\begin{frame}[t]
	\begin{block}{}
		Así pues, \(y_1,y_2\) conforman un c.f.s y la solución general de la E.D. está dada por
		\[
			y(x) = c_1e^{- \frac{b}{2a} x} +c_2xe^{- \frac{b}{2a}}.
		\]
	\end{block} \vspace{6mm}
	\begin{block}{Caso 3. Raíces Imaginarias \(m_1= \alpha +i \beta\), \(m_2= \alpha +i \beta\).}
		En este caso, se tienen soluciones complejas \(y_1(x) = e^{(\alpha +i \beta) x}\), \(y_2(x) = e^{(\alpha - i \beta) x}\). \\[2mm]
		Observemos lo siguiente
		\[
			y_1(x) = e^{(\alpha +i \beta) x} = e^{\alpha x+i \beta x} = e^{\alpha x} \cdot e^{i \beta x}.
		\]
		Usemos la fórmula de Euler,
		\begin{center}
			\color{red} \underline{\color{black} \(e^{i \theta} = \cos \theta + i \sin \theta\)} 
		\end{center} 
	\end{block}
\end{frame}

\begin{frame}[t]
	\begin{block}{}
		Luego:
		\[
			\begin{array}{rcl}
				y_1(x) & = & e^{\alpha x} \cdot \big(\cos \beta x+ i \sin \beta x\big) \\[2mm]
				y_1(x) & = & \color{red} \underbrace{\color{black} e^{\alpha x} \cos \beta x} _{Re(y_1(x))} \color{black} +i \color{red} \underbrace{\color{black} e^{\alpha x} \sin \beta x} _{Im(y_1(x))} .
			\end{array}
		\]
		De manera análoga, se obtiene que:
		\[
			y_2(x) = e^{\alpha x} \cos \beta x - i e^{\alpha x} \sin \beta x.
		\]
		Ahora, si consideramos la suma de \(y_1(x)\) y \(y_2(x)\) y multiplicamos por \(1/2\) obtenemos que
		\[
			\tilde{y_1} (x) = e^{\alpha x} \cos \beta x \hspace{3mm} \mbox{ es solución de la E.D.}
		\]
		Ahora, si restamos a \(y_2\) de \(y_1\) y lo multiplicamos por \(-1/2\) se obtiene otra solución:
	\end{block}
\end{frame}

\begin{frame}[t]
	\begin{block}{}
		\[
			\tilde{y_2} (x) = e^{\alpha x} \sin \beta x.
		\]
		Veamos que \(\tilde{y_1} , \tilde{y_2}\) sol L.I. \vspace{-6mm}
		\[
			\begin{array}{rcl}
				W[y_1,y_2] (x) & = & \begin{vmatrix}
					e^{\alpha x} \cos \beta x & e^{\alpha x} \sin \beta x \\
					-e^{\alpha x} \sin \beta x & e^{\alpha x} \cos \beta x
				\end{vmatrix}\\[4mm]
				& = & e^{2\alpha x} (\cos ^2x+ \sin ^2x) \\[2mm]
				& = & e^{2\alpha x} \ne 0, \hspace{5mm} \forall x. \vspace{-3mm}
			\end{array}
		\]
		Es fácil verificar que \(\tilde{y_1} \;,\; \tilde{y_2}\) son L.I. usando el wronskiano.
		\[
			\tilde{y_1} = e^{\alpha x} \cos \beta x \mbox{  y  } \tilde{y_2} (x) = e^{\alpha x} \sin \beta x.
		\]
		Luego, \(\tilde{y_1}\) y \(\tilde{y_2}\) conforman un c.f.s para la E.D. Por consiguiente, la solución general está dada por:
		\[
			\begin{array}{c}
				y(x) = c_1 e^{\alpha x} \cos \beta x+c_2e^{\alpha x} \sin \beta x. \\[2mm]
				\alpha = Re(m) \;,\; \beta =Im(m) \hspace{5mm} m = \alpha +i \beta .
			\end{array}
		\]
	\end{block}
\end{frame}

\begin{frame}[t]
	\begin{example}
		Determine la solución general de las siguientes E.D.L.H. con coeficientes constantes de \(2^{do}\) orden.
		\begin{enumerate}
			\item \(y'' -10y' +25y=0\).
			\item \(2y'' -3y' +4y=0\).
			\item \(y'' +4y' -2y=0\).
		\end{enumerate}
	\end{example}
\end{frame}
\begin{frame}[t]
\end{frame}

\begin{frame}[t]
	\begin{alertblock}{Ejercicio.}
		Determine la solución del P.V.I. \(4y'' +4y' +17y=0\), con \(y(0) =1\), y \(y'(0) =2\). Además, esboce la gráfica de la curva solución.
	\end{alertblock}
\end{frame}
\begin{frame}[t]
\end{frame}
% )))

\section{E.D.L.H. de Orden Superior.} % (((
\begin{frame}[t]
	\frametitle{E.D.L.H. de Orden Superior.}
	\begin{block}{}
		Para resolver una E.D. de orden \(n\) de la forma
		\[
			a_ny^{(n)} +a_{n-1} y^{(n-1)} + \;\ldots\; + a_1y' +a_0y =0.
		\]
		donde \(a_i\), \(i=1, \;\ldots,\; n\), son constantes reales y \(a_n \ne 0\). \\[2mm]
		Se debe resolver una ecuación polinomial de la forma:
		\[
			\underbrace{a_nm^n+a_{n-1} m^{n-1} + \;\ldots\; + a_1m+a_0=0} _\text{Ecuación Característica.}
		\]
		\begin{enumerate}
			\item Si todas las raíces de la ecuación son reales y distintos, la solución de la ecuación será:
				\[
					y(x) = c_1e^{m_1x} + c_2e^{m_2x} + \;\cdots\; + c_ne^{m_nx},
				\]
				donde \(c_1, \;\ldots,\; c_n\) son cts. arb. y \(m_1, \;\ldots,\; m_n\) son las raíces de la ec. característica.
		\end{enumerate}
	\end{block}
\end{frame}

\begin{frame}[t]
	\begin{block}{}
		\begin{enumerate}
			\setcounter{enumi}{1}
		\item Cuando la raíz \(m_j\) tiene multiplicidad \(k\), la solución general de la ecuación diferencial debe contente la combinación lineal siguiente
			\[
				c_1e^{m_jx} + c_2xe^{m_jx} +c_3x^2e^{m_jx} + \;\cdots\; + c_{k} x^{k-1} e^{m_jx}.
			\]
		\item La ecuación característica tiene raíces complejas. Puesto que las raíces complejas aparecen en pares, la solución general deberá contener lo siguiente.
			\begin{itemize}
				\item Si la multiplicidad de las raíces \(m_1= \alpha +i \beta\), \(m_2= \alpha -i \beta\), es \(1\):
					\[
						c_1e^{\alpha x} \cos \beta x+ c_2 e^{\alpha x} \sin \beta x.
					\]
				\item Si la multiplicidad de las raíces \(m_1= \alpha +i \beta\), \(m_2 = \alpha - i \beta\), es \(k\):
					\[
						\begin{array}{c}
							\hspace{-1cm} c_1e^{\alpha x} \cos \beta x + c_2e^{\alpha x} \sin \beta x + c_3xe^{\alpha x} \cos \beta x + c_4xe^{\alpha x} \sin \beta + \;\cdots\; + \\[2mm]
							+ \;\cdots\; + c_{2k-1} x^{k-1} e^{\alpha x} \cos \beta x+c_{2k} x^{k-1} e^{\alpha x} \sin \beta x.
						\end{array}
					\]
			\end{itemize}
		\end{enumerate}
	\end{block}
\end{frame}

\begin{frame}[t]
	\begin{example}
		Determine la solución general de las siguientes E.D.L.H.
		\begin{enumerate}
			\item \(\dfrac{d^4y}{dx^4} +4 \dfrac{d^2y}{dx^2} +4y=0\).
			\item \(y''' + 3y'' -4y=0\).
		\end{enumerate}
	\end{example}
\end{frame}
% )))

\end{document}
