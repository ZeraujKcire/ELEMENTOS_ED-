\documentclass{beamer}

% === PAQUETES === (((
\usepackage[utf8]{inputenc}
\usepackage{amsfonts}
\usepackage{amsmath}
\usepackage{graphicx}
\usepackage{anysize} 
% )))

% === DATOS === (((
% \pdfinfo{%
	% /Title    (Tarea <++> de Ecuaciones Diferenciales)
	% /Author   (Sandra Elizabeth Delgadillo Alemán)
% }
\marginsize{1cm}{1cm}{1cm}{1cm} 
\pagestyle{empty}
% )))

% === TITULO === (((
\newcommand{\titulo}{
	\begin{minipage}{0.25\linewidth}
		\includegraphics[width= 0.9 \linewidth]{IMAGENES/log23.png}
	\end{minipage}
	\begin{minipage}{0.75\linewidth}
		\begin{center}
			\bfseries
			CENTRO DE CIENCIAS BÁSICAS \\
			DEPARTAMENTO DE MATEMÁTICAS Y FÍSICA \\
			ACADEMIA DE MATEMÁTICA AVANZADAS
		\end{center}
	\end{minipage}\\
	\begin{table}[ht]
		\centering
		\begin{tabular}{|*{3}{l|}p{3cm}|}
			\hline
			\textbf{Nombre del Estudiante:} & & \textbf{Fecha:} & \\ \hline
			\textbf{Materia:} & Ecuaciones Diferenciales & \textbf{Carrera:} &  \\ \hline
			\textbf{Profesor:} & Sandra Elizabeth Delgadillo Alemás & \textbf{Semestre:} & \\ \hline
			\textbf{Periodo:} & () Enero--Junio () Agosto--Diciembre & & \\ \hline
			\textbf{Tipo de Examen:} & Parcial: 1() \hspace{2mm} 2() \hspace{2mm} 3() & \textbf{Calificación:} & \\ \hline
		\end{tabular}
	\end{table}
} 
% )))

\begin{document}

\frame{\titlepage}

\section{Movimiento Vibratorio de Sistemas Mecánicos.} % (((
\begin{frame}[t]
	\frametitle{Movimiento Vibratorio de Sistemas Mecánicos.}
	Suponga una masa \(m\) unida a un resorte flexible colgado sobre un soporte rígido. La distancia de alargamiento dependerá de la masa. Según la Ley de Hooke, el resorte mismo ejerce una fuerza de restitución opuesta a la dirección de alargamiento y proporcional al dicho alargamiento \(S\), es decir:
	\[
		F=kS \;,\; k = \mbox{constante del resorte.}
	\]
	\begin{figure}[ht]
		\centering
		\includegraphics[width= 0.8 \linewidth]{IMAGENES/1/tikz.pdf}
	\end{figure}
\end{frame}

\begin{frame}[t]
	\begin{columns}
		\column{0.3\textwidth}
		Equilibrio
		\[
			\begin{array}{rcl}
				F_g-F_R & = & 0 \\
				F_g-kS & = & 0 \\
				\iff mg = kS.
			\end{array}
		\]
		\column{0.3\textwidth}
		Movimiento
		\[
			\begin{array}{rcl}
				F_T & = & F_g-F_R \\
				F_T & = & mg - k(S+x) \\
				F_T & = & mg - kS-kx \\
				F_T & = & -kx.
			\end{array}
		\]
	\end{columns}
\end{frame}

\begin{frame}[t]
	\begin{block}{}
		Por la \(2^{da}\) Ley de Newton \(F_T = ma = m \dfrac{dv}{dt} = m \dfrac{d^2x}{dt^2}\). Luego, igualando \(F_T = m \dfrac{d^2x}{dt^2} = -kx\), \(k,m >0\),
		\[
			\iff m \dfrac{d^2x}{dt^2} +kx =0 \iff \dfrac{d^2x}{dt^2} + \underbrace{\dfrac{k}{m}} _{w^2 = k/m >0}  x =0
		\]
		Luego, la ecuación que describe el movimiento de \(m\) está dada por
		\begin{center}
			\color{red} \fbox{\color{black} \(\dfrac{d^2x}{dt^2} + w^2x =0\)} \\
			\footnotesize E.D. para describir el Movimiento Armónico Simple.
		\end{center} 
		Resolvemos la E.D. usando la ecuación característica \(m^2+w^2=0 \iff m^2=-w^2 \iff m_{1,2} = \pm iw\), \(\alpha =0\), \(\beta =w\).
	\end{block}
\end{frame}
% )))

\end{document}
