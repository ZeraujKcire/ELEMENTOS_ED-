\documentclass{beamer}

% === PAQUETES === (((
\usepackage[utf8]{inputenc}
\usepackage{amsfonts}
\usepackage{amsmath}
\usepackage{graphicx}
\usepackage{anysize} 
% )))

% === DATOS === (((
% \pdfinfo{%
	% /Title    (Tarea <++> de Ecuaciones Diferenciales)
	% /Author   (Sandra Elizabeth Delgadillo Alemán)
% }
\marginsize{1cm}{1cm}{1cm}{1cm} 
\pagestyle{empty}
% )))

% === TITULO === (((
\newcommand{\titulo}{
	\begin{minipage}{0.25\linewidth}
		\includegraphics[width= 0.9 \linewidth]{IMAGENES/log23.png}
	\end{minipage}
	\begin{minipage}{0.75\linewidth}
		\begin{center}
			\bfseries
			CENTRO DE CIENCIAS BÁSICAS \\
			DEPARTAMENTO DE MATEMÁTICAS Y FÍSICA \\
			ACADEMIA DE MATEMÁTICA AVANZADAS
		\end{center}
	\end{minipage}\\
	\begin{table}[ht]
		\centering
		\begin{tabular}{|*{3}{l|}p{3cm}|}
			\hline
			\textbf{Nombre del Estudiante:} & & \textbf{Fecha:} & \\ \hline
			\textbf{Materia:} & Ecuaciones Diferenciales & \textbf{Carrera:} &  \\ \hline
			\textbf{Profesor:} & Sandra Elizabeth Delgadillo Alemás & \textbf{Semestre:} & \\ \hline
			\textbf{Periodo:} & () Enero--Junio () Agosto--Diciembre & & \\ \hline
			\textbf{Tipo de Examen:} & Parcial: 1() \hspace{2mm} 2() \hspace{2mm} 3() & \textbf{Calificación:} & \\ \hline
		\end{tabular}
	\end{table}
} 
% )))

\begin{document}

\frame{\titlepage}

\section{Movimiento Vibratorio de Sistemas Mecánicos.} % (((
\begin{frame}[t]
	\frametitle{Movimiento Vibratorio de Sistemas Mecánicos.}
	Suponga una masa \(m\) unida a un resorte flexible colgado sobre un soporte rígido. La distancia de alargamiento dependerá de la masa. Según la Ley de Hooke, el resorte mismo ejerce una fuerza de restitución opuesta a la dirección de alargamiento y proporcional al dicho alargamiento \(S\), es decir:
	\[
		F=kS \;,\; k = \mbox{constante del resorte.}
	\]
	\begin{figure}[ht]
		\centering
		\includegraphics[width= 0.8 \linewidth]{IMAGENES/1/tikz.pdf}
	\end{figure}
\end{frame}

\begin{frame}[t]
	\begin{columns}
		\column{0.3\textwidth}
		Equilibrio
		\[
			\begin{array}{rcl}
				F_g-F_R & = & 0 \\
				F_g-kS & = & 0 \\
				\iff mg = kS.
			\end{array}
		\]
		\column{0.3\textwidth}
		Movimiento
		\[
			\begin{array}{rcl}
				F_T & = & F_g-F_R \\
				F_T & = & mg - k(S+x) \\
				F_T & = & mg - kS-kx \\
				F_T & = & -kx.
			\end{array}
		\]
	\end{columns}
\end{frame}

\begin{frame}[t]
	\begin{block}{}
		Por la \(2^{da}\) Ley de Newton \(F_T = ma = m \dfrac{dv}{dt} = m \dfrac{d^2x}{dt^2}\). Luego, igualando \(F_T = m \dfrac{d^2x}{dt^2} = -kx\), \(k,m >0\),
		\[
			\iff m \dfrac{d^2x}{dt^2} +kx =0 \iff \dfrac{d^2x}{dt^2} + \underbrace{\dfrac{k}{m}} _{w^2 = k/m >0}  x =0
		\]
		Luego, la ecuación que describe el movimiento de \(m\) está dada por
		\begin{center}
			\color{red} \fbox{\color{black} \(\dfrac{d^2x}{dt^2} + w^2x =0\)} \\
			\footnotesize E.D. para describir el Movimiento Armónico Simple.
		\end{center} 
		Resolvemos la E.D. usando la ecuación característica \(m^2+w^2=0 \iff m^2=-w^2 \iff m_{1,2} = \pm iw\), \(\alpha =0\), \(\beta =w\).
	\end{block}
\end{frame}

\begin{frame}[t]
	\begin{block}{}
		Luego, la solución general de la E.D. de movimiento es: \(x(t) = c_1 \cos t + c_2 \sin wt\). \\[2mm]
		Establecemos las condiciones iniciales como \(x(0) = x_0\), \(x' (0) = x_1\), aplicando las CI's.
		\[
			\begin{array}{rcl}
				x' (t) & = & -w c_1 \sin w t -w c_2 \cos w t \\[2mm]
				x(0) & = & c_1 \cos (0)+c_2 \sin (0) = x_0 \\[2mm]
				x'(0) & = & -wc_1 \sin w(0)+wc_2 \cos w(0) = x_1
			\end{array}
		\]
		\begin{center}
			\(\therefore \hspace{5mm}\) \color{red} \fbox{\color{black} \( x(t) = x_0 \cos wt+ \dfrac{x_1}{w} \sin wt\)}\\
			\hspace{5mm} \footnotesize Ecuación de movimiento armónico simple. \\[2mm]
			\color{black} \normalsize \(x(t)= A \sin (wt+ \phi ) \hspace{1cm} A = \sqrt{x_0^2+(x_1/w) ^2}\).\\
			\color{red} \hspace{4cm} \footnotesize Amplitud.
		\end{center}
	\end{block}
\end{frame}

\begin{frame}[t]
	\begin{block}{}
		\[
			\phi = \tan ^{-1} \bigg(\dfrac{x_0w}{x_1}\bigg) \hspace{4mm} T = \dfrac{2 \pi}{w} \;,\;  [T] = seg \hspace{4mm} f = \dfrac{w}{2 \pi} \;,\; [f] = 1/s = Hz.
		\]
		\color{red} \hspace{1cm} \footnotesize Angulo de fase \hspace{1.3cm} Periodo \hspace{1.5cm} Frecuencia.
		\begin{figure}[hbt!]
			\centering
			\includegraphics[width= 0.5 \linewidth]{IMAGENES/2/tikz.pdf}
		\end{figure}
	\end{block}
\end{frame}

\begin{frame}[t]
	\begin{example}
		Una masa que pesa \(2lb\) hace que un resorte se estire \(6\;in\).
		Cuando \(t=0\)  la masa se suelta desde un punto a \(8i\;n\) abajo de la posición de equilibrio con una velocidad hacia arriba de \(4/3\;ft/s\).
		Deduzca la ecuación de movimiento libre.
	\end{example}
\end{frame}
% )))

\end{document}
