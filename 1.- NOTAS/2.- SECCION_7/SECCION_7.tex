\documentclass{beamer}


% === PAQUETES === (((
\usepackage[utf8]{inputenc}
\usepackage{amsfonts}
\usepackage{amsmath}
\usepackage{graphicx}
\usepackage{anysize} 
% )))

% === DATOS === (((
% \pdfinfo{%
	% /Title    (Tarea <++> de Ecuaciones Diferenciales)
	% /Author   (Sandra Elizabeth Delgadillo Alemán)
% }
\marginsize{1cm}{1cm}{1cm}{1cm} 
\pagestyle{empty}
% )))

% === TITULO === (((
\newcommand{\titulo}{
	\begin{minipage}{0.25\linewidth}
		\includegraphics[width= 0.9 \linewidth]{IMAGENES/log23.png}
	\end{minipage}
	\begin{minipage}{0.75\linewidth}
		\begin{center}
			\bfseries
			CENTRO DE CIENCIAS BÁSICAS \\
			DEPARTAMENTO DE MATEMÁTICAS Y FÍSICA \\
			ACADEMIA DE MATEMÁTICA AVANZADAS
		\end{center}
	\end{minipage}\\
	\begin{table}[ht]
		\centering
		\begin{tabular}{|*{3}{l|}p{3cm}|}
			\hline
			\textbf{Nombre del Estudiante:} & & \textbf{Fecha:} & \\ \hline
			\textbf{Materia:} & Ecuaciones Diferenciales & \textbf{Carrera:} &  \\ \hline
			\textbf{Profesor:} & Sandra Elizabeth Delgadillo Alemás & \textbf{Semestre:} & \\ \hline
			\textbf{Periodo:} & () Enero--Junio () Agosto--Diciembre & & \\ \hline
			\textbf{Tipo de Examen:} & Parcial: 1() \hspace{2mm} 2() \hspace{2mm} 3() & \textbf{Calificación:} & \\ \hline
		\end{tabular}
	\end{table}
} 
% )))


\begin{document}
\frame{\titlepage}

\section{7. Ecuaciones de Variables Searables.}

\begin{frame}[t]
	\frametitle{7. Ecuaciones de Variables Searables.}
	\vspace{-4mm}
	\begin{definition}
		Se dice que una ecuación diferencial es \textbf{de variables separables} si tiene la siguiente forma.
		\[
			\dfrac{dy}{dx} = f(x,y) = g(x) \cdot h(y).
		\]
	\end{definition}
	\begin{example}
		Identifique si las siguientes ecuaciones diferenciales son de variables separables.
		\begin{enumerate}
			\item \(\dfrac{dy}{dx} = x^2y^2e^{3x +4y}\). \\
				\textbf{Solución.} \(\dfrac{dy}{dx} =x^2y^2e^{3x+4y} =x^2y^2e^{3x} e^{4y} = \underbrace{(x^2e^{3x})} _{g(x)} \underbrace{(y^2e^{4y})}_{h(y)}\).
				E.D. de variables separables.
		\end{enumerate}
	\end{example}
\end{frame}

\begin{frame}[t]
	\begin{exampleblock}{}
		\begin{enumerate}
			\setcounter{enumi}{1}
		\item \(\dot{x} =1+xy\). \textbf{Solución.} No es E.D. de variables separables.
		\item \(y \,' =y+ \sin x\). \textbf{Solución.} No es E.D. de variables separables.
		\item \(dye^xy+(x+xy^2)dx=0 \). \\
			\textbf{Solución.} 
				\begin{columns}[t]
					\column{0.4\textwidth}
					\[
						\begin{array}{rcl}
							- dye^xy & = & (x+xy^2) dx \\[2mm]
							\dfrac{dy}{dx} & = & - \dfrac{x(1+y^2)}{e^xy} \\[2mm]
							\dfrac{dy}{dx} & = & \underbrace{\Bigg(\dfrac{-x}{e^x}\Bigg)}_{g(x)} \underbrace{\Bigg(\dfrac{1+y^2}{y}\Bigg)}_{h(y)}.
						\end{array}
					\]
					\column{0.5\textwidth}
					Si \(y=y(x)\) es equivalente:\\ \((x+xy^2) +e^xy \dfrac{dy}{dx} =0\).
					\[
						\begin{array}{rcl}
							\dfrac{dy}{dx} & = & \dfrac{-(x+xy^2)}{e^xy} \\[2mm]
							& = & \dfrac{-x(1+y^2)}{e^xy} \\[2mm]
							& = & \Bigg(\dfrac{-x}{e^x}\Bigg) \Bigg(\dfrac{1+y^2}{y}\Bigg)
						\end{array}
					\]
					\(\therefore \hspace{2mm}\) Es una E.D. de variables separables.
				\end{columns}
		\end{enumerate}
	\end{exampleblock}
\end{frame}

\begin{frame}[t]
	\begin{block}{\vspace*{-2ex}}
		La E.D. de variables separables más sencilla que existe es:
		\begin{center}
			\textcolor{magenta!70}{\fbox{\textcolor{black}{\(\dfrac{dy}{dx} =g(x) \cdot 1\)}}} \hspace{5mm}\textcolor{magenta!70}{\textit{i.e.} \(\dfrac{dy}{dx} =g(x)\)}
		\end{center} 
		Para obtener la función \(y=y(x)\) se debe integrar de ambos lados:
		\[
			\int \dfrac{dy}{dx} dx = \int g(x) dx.
		\]
		Supongamos que existe la antiderivada de \(g(x)\), es decir, \(G(x)\), tal que \(G \,'(x) =g(x)\).\\
		Luego sustituyendo \(g(x)\) por su antiderivada \(G \,' (x)\), se tiene
		\[
			\begin{array}{c}
				y(x) = \dis\int G \,' (x) dx = \dis\int \dfrac{d}{dx} G(x) dx = G(x) +c. \\[5mm]
				\iff y(x) =G(x) +c. \hspace{1cm}\mbox{\alert{Solución gral. explícita.}}
			\end{array}
		\]
	\end{block}
\end{frame}

\begin{frame}[t]
	\begin{example}
		Considere la E.D. \(\dfrac{dy}{dx} =1+e^{2x}\) y resuélvala. \\[2mm]
		\textbf{Solución.} Integramos de ambos lados:
		\[
			\begin{array}{c}
				\dis\int \dfrac{dy}{dx} dx= \dis\int (1+e^{2x}) dx = \dis\int dx+ \dis\int e^{2x} dx. \\[5mm]
				\iff \underbrace{y(x) = x+ \dfrac{1}{2} e^{2x} +c} \hspace{5mm} \mbox{Sol. gral. explícita de la E.D.}
			\end{array}
		\]
	\end{example}
	Consideremos nuevamente la E.D. de variables separables para ilustrar el método de solución
	\[
		\dfrac{dy}{dx} = g(x) h(y) \;\;, \hspace{5mm} \textcolor{red}{h(y) \ne 0.}
	\]
	dividimos \(h(y)\) a ambos lados de la E.D. \(\dfrac{1}{h(y)} \dfrac{dy}{dx} = g(x)\).
\end{frame}

\begin{frame}[t]
	Hagamos \(p(y) = \dfrac{1}{h(y)}\) en la E.D. anterior,
	\[
		p(y) \cdot \dfrac{dy}{dx} = g(x).
	\]
	integramos de ambos lados con respecto a \(x\).
	\[
		\dis\int p(y) \dfrac{dy}{dx} dx= \dis\int g(x) dx.
	\]
	Supongamos que existen las antiderivadas de \(p(y)\) y \(g(x)\). Es decir, que existen \(P(y)\) tt \(G(x)\) tales que 
	\[
		P \,' (y) =p(y) \hspace{5mm} G \,' (x) =g(x).
	\]
	Luego, sustituyendo se tiene:
	\[
		\dis\int P \,' (y) dy= \dis\int P \,' (y) \dfrac{dy}{dx} dx = \dis\int g \,' (x) dx.
	\]
	Observemos que \(\dfrac{d}{dx} P(y) =P \,' (y) \dfrac{dy}{dx}\). \\[2mm]
\end{frame}

\begin{frame}[t]
	Sustituimos en la ecuación antrerior,
	\[
		\begin{array}{rcl}
			\dis\int \dfrac{d}{dx} (P(y)) dx &=& \dis\int \dfrac{d}{dx} (G(x)) dx \\[5mm]
			\iff P(y) +c_1 & = & G(x) +c_2 \\[2mm]
			\iff P(y) -G(x) & = & c_2-c_1 \\[2mm]
		\end{array}
	\]
	\begin{center}
		\hspace{-9mm}\textcolor{magenta!70}{\fbox{\textcolor{black}{\(P(y) -G(x) \hspace{2mm} = \hspace{2mm} c\)}}} \hspace{5mm}\\[2mm]
		\textcolor{magenta!70}{Solución General Implícita}.
	\end{center} 
	Ahora, bien, consideremos la E.D. \(\dfrac{dy}{dx} =g(t (h(t)))\) sujeta a una C.I. dada por \(y(t_0) =y_0\), \(t_0 \in I\).\\[2mm]
	Este P.V.I. se puede resolver de dos maneras.
\end{frame}

\begin{frame}[t]
	\begin{enumerate}
		\item Determinar la solución general de la E.D. de variables separables y luego encontrar el valor de la cte. arbitraria \(C\) de tal forma que satisfaga la condición inicial \(y(t_0) =y_0\).
		\item Otra manera de determinar la solución es integrando de \(t_0\) a \(t\) la ecuación estrella
			\begin{align*}
				\color{magenta!70} \dis\int _{t_0} ^t \dfrac{d}{ds} \big(P(y(s))\big) ds & \color{magenta!70} = \color{magenta!70} \dis\int _{t_0} ^tG(s) ds \\[2mm]
				\iff P(y(s)) \Big| _{t_0} ^t & = G(s) \Big| _{t_0} ^t \\[2mm]
				\iff P(y(t)) - P(y(t_0)) & = G(t) -G(t_0) \\[2mm]
				\iff \textcolor{red}{\underline{\mathstrut \textcolor{black}{P(y(t)) -G(t)}}} & \textcolor{red}{\underline{\mathstrut \textcolor{black}{= P(y(t_0)) -G(t_0)}}} 
			\end{align*}
			Así, \(P(y(t) -G(t) =P(y(t_0))) -G(t_0)\) es la sol. particular implícita de la E.D.
	\end{enumerate}
\end{frame}

\begin{frame}[t]
	\begin{example}
		Determine la solución de la E.D. \((t+1) e^y \cdot  \dfrac{dy}{dx} -(t-1) =0\). que satisface la C.I. \(y(1) =2\).\\[2mm]
		\textbf{Solución.} La E.D. en su forma normal es:
		\[
			\begin{array}{rcl}
				\dfrac{dy}{dt} = \dfrac{t-1}{e^y(t+1)} &=& \underbrace{\Big(\dfrac{t-1}{t+1}\Big)} _{g(t)} \underbrace{e^{-y}}_{h(y)} \\[2mm]
				\iff \hspace{5mm} e^y \dfrac{dy}{dx} & = & \dfrac{t-1}{t+1}
			\end{array}
		\]
		Integramos de ambos lados con respecto a \(t\),
		\begin{equation}
			\begin{array}{rcl}
				\dis\int e^ydy = \dis\int e^y \dfrac{dy}{dt} dt & = & \dis\int \dfrac{t-1}{t+1} dt \\[2mm]
				e^y & = & \dis\int \dfrac{t-1}{t+1} dt \label{estrella}
			\end{array}
		\end{equation}
	\end{example}
\end{frame}

\begin{frame}[t]
	\vspace{-4mm}
	\begin{exampleblock}{}
		\[
			\begin{array}{rcl}
				\dis\int \dfrac{t-1}{t+1} dt & = & \dis\int \dfrac{t-1+1-1}{t-1} dt \\[4mm]
				& = & \dis\int \dfrac{(t+1) -2}{t+1} dt \\[4mm]
				& = & \dis\int \Bigg[1- \dfrac{2}{t+1}\Bigg] dt \\[4mm]
				& = & t-2\ln(t+1) +c
			\end{array}
		\]
		Por lo tanto, la ec. (\ref{estrella}) es equivalente a:
		\[
			\begin{array}{c}
				\underline{e^y = t-\ln(t+1) ^2+c} \;\;,\hspace{5mm} y \in (-1, \infty) \\[2mm]
				\mbox{Solución gral. implícita de la E.D.}
			\end{array}
		\]
		Despejamos y aplicando \(\ln\) de ambos lados
		\[
			\begin{array}{rcl}
				\ln e^y & = & \ln (t- \ln (t+1) ^2+c) \\[2mm]
				y(t) & = & \ln (t- \ln (t+1) ^2+c), \hspace{5mm} \mbox{Sol. gral. explícita de la E.D.}
			\end{array}
		\]
	\end{exampleblock}
\end{frame}

\begin{frame}[t]
	\begin{exampleblock}{}
		Ahora, determinemos el valor de \(c\) de tal forma que \(y=2\), en \(t=1\). \\[2mm]
		Sustituimos \(t=1\), y \(y=2\) en la sol. gral. implícita,
		\[
			\begin{array}{rcl}
				e^2 & = & 1- \ln 4+c \\[2mm]
				c & = & e^2+ \ln 4-1 \approx 7.77\\[2mm]
				\therefore \hspace{4mm} y(t) & = & \ln (t- \ln (t+1) ^2+e^2-1+ \ln 4) \hspace{4mm} \mbox{\textcolor{red}{Sol. del P.V.I}}
			\end{array}
		\]
	\end{exampleblock}
\setbeamercolor*{block title example}{fg=white,bg= magenta!70}
\setbeamercolor*{block body example}{fg= magenta,bg= magenta!5}
	\vspace{-2mm}
	\begin{exampleblock}{Forma alternativa}
		Otra manera de resolver el P.V.I.
		\[
			\begin{array}{rcl}
				\color{blue!70} \dfrac{dy}{dt} & \color{blue!70} = & \color{blue!70} \dfrac{t-1}{(t+1) e^y} \\[5mm]
				\iff \hspace{2mm} e^y \dfrac{dy}{dt} & = & \dfrac{t-1}{t+1}
			\end{array}
		\]
		Integramos de \(1\) a \(t\).
	\end{exampleblock}
\end{frame}

\begin{frame}[t]
\setbeamercolor*{block title example}{fg=white,bg= magenta!70}
\setbeamercolor*{block body example}{fg= magenta,bg= magenta!5}
	\vspace{-3mm}
	\begin{exampleblock}{}
		\[
			\begin{array}{rcl}
				\dis\int _2^ye^rdr\dis\int _1^te^y \dfrac{dy}{ds} ds & = & \dis\int _1^t \dfrac{s-1}{s+1} ds \\[5mm]
				\iff \hspace{2mm} e^r \Big|_2^y & = & \dis\int _1^t \Bigg(1- \dfrac{2}{s+1}\Bigg) ds\\[5mm]
				\iff \hspace{2mm} e ^r\Big|_2^y & = & \dis\int _1^tds-2 \dis\int _1^t \dfrac{ds}{s+1} \\[5mm]
				\iff \hspace{2mm} e^y-e^2 & = & s-2 \ln \big| s+ 1 \big| \Big| _1^t \\[5mm]
				\iff \hspace{2mm} e^y & = & t-2 \ln \big| t+1 \big| -1+ \ln 4+e^2
			\end{array}
		\]
		Aplicamos \(\ln\) de ambos lados,
		\[
			\begin{array}{rcl}
				\ln (e^y) & = & \ln \big(t-2 \ln \big| t+1 \big| -1+ \ln 4+c^2\big) \\[2mm]
				\iff y & = & \ln \Bigg(t+ \ln \Big(\dfrac{4}{(t+1) ^2} \Big)-1+e^2\Bigg) \hspace{3mm} \mbox{\textcolor{red}{sol. de P.V.I}}
			\end{array}
		\]
	\end{exampleblock}
\end{frame}

\begin{frame}[t]
	\begin{example}
		Resuelve la E.D. \((x+xy^2) dx+e^xydy =0\). \\[2mm]
		\textbf{Solución.} La E.D. es equivalente a la siguiente 
		\[
			\begin{array}{rcl}
				x(1+y^2) dx+e^xydy & = & 0 \\[2mm]
				\iff e^xydy & = & -x(1+y^2) dx \\[2mm]
				\iff \dfrac{y}{1+y^2} dy & = & - \dfrac{x}{e^x} dx
			\end{array}
		\]
		Integramos de ambos lados, \vspace{-2mm}
		\[
			\begin{array}{rrcl}
				&\dis\int \dfrac{y}{1+y^2} dy & = & \dis\int - \dfrac{x}{e^x} dx \\[2mm]
				\iff & \dfrac{1}{2} \ln \big| 1+y^2 \big| & = & xe^{-x} - \dis\int e^{-x} dx \\[2mm]
				\iff & \dfrac{1}{2} \ln \big| 1+y^2 \big| & = & xe^{-x} +e^{-x} +c,\\
				&&&  \hspace{2mm} \mbox{sol. gral. implícita de la E.D.}
			\end{array}
		\]
	\end{example}
\end{frame}

\begin{frame}[t]
	\begin{exampleblock}{}
		Despejamos \(y\), y multiplicamos por \(2\),
		\[
			\begin{array}{rcl}
				\ln (1+y^2) & = & 2xe^{-x} +2e^{-x} +c_1 \\[2mm]
				e^{\ln (1+y^2)} & = & e^{2xe^{-x} +2e^{-x} +c_1} \\[2mm]
				1+y^2 & = & e^{2xe^{-x}} e^{2e^{-x}} e^{c_1}
			\end{array}
		\]
		Por lo tanto, la sol. explícita de la E.D. es
		\[
			\begin{array}{rcl}
				y & = & \pm \sqrt{c_2e^{2xe^{-x}} e^{2e^{-x}} -1} , \hspace{5mm} c_2 >0\\[2mm]
				y(x) & = & \pm \sqrt{c_2e^{2e^{-x} (x+1)} -1}, \hspace{5mm}c_2 >0
			\end{array}
		\]
	\end{exampleblock}
\end{frame}

\begin{frame}[t]
	\vspace{-3mm}
	\begin{alertblock}{Ejercicio.}
		Resuelve la siguiente E.D. y P.V.I.
		\begin{enumerate}
			\item \(\dfrac{dy}{dx} =1-x+y^2-xy^2\).
			\item \((e^{2y} -y) \cos xdy -e^y \sin 2xdx=0\).
		\end{enumerate}
	\end{alertblock}
	\begin{enumerate}
		\item \textbf{Solución.} La E.D. es equivalente a la siguiente:
			\[
				\dfrac{dy}{(1+y^2)}  =  (1-x) dx
			\]
			Integrando ambos lados
			\[
				\begin{array}{rcl}
					 \dis\int \dfrac{dy}{1+y^2} & = & \dis\int (1-x) \\[2mm]
					 \arctan y & = & x- \dfrac{x^2}{2} +c \;\;, \hspace{5mm} \mbox{solución gral. implícita}\\
					\mbox{Despejando } y&&\\[-2mm]
					y & = & \tan \Bigg(x- \dfrac{x^2}{2} +c\Bigg) \;\;, \hspace{5mm} \mbox{solución gral. implícita}
				\end{array}
			\]
	\end{enumerate}
\end{frame}

\begin{frame}[t]
	\begin{enumerate}
			\setcounter{enumi}{1}
		\item \textbf{Solución.} La E.D. es equivalente a la siguiente:
			\[
				\begin{array}{rcl}
					\dfrac{e^{2y} -y}{e^y} dy & = & \dfrac{2 \sin x \cos x dx}{\cos x} \\[2mm]
					\Bigg(e^y- \dfrac{y}{e^y}\Bigg) dy & = & 2 \sin x dx
				\end{array}
			\]
			Integrando en ambos lados
			\[
				\begin{array}{rcl}
					\dis\int \Bigg(e^y- \dfrac{y}{e^y}\Bigg) dy & = & 2 \dis\int \sin xdx \\[5mm]
					\iff e^y+ye^{-y}+ e^{-y} & = & -2 \cos x+c\;\;,
				\end{array}
			\]
			Es la solución gral. implícita de la E.D.\\
			Determinemos el valor de \(c\) que cumpla con el P.V.I.\\
			Sustituimos \(y=0\), y \(x=0\)
			\[
				\begin{array}{rcl}
					e^0+0+e^0&=&-2(1) +c \\[2mm]
					c & = & 4 \\[2mm]
				\end{array}
			\]
	\end{enumerate}
\end{frame}

\begin{frame}[t]
	\[
		\therefore \hspace{5mm} e^y+ye^{-y} +e^{-y} = -2 \cos x+4.
	\]
	es la solución particular implícita que cumple con la condición inicial dada.
\end{frame}

\end{document}
